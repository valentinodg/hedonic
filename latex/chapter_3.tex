\chapter{Implementazione}

In questo capitolo andremo a delineare e descrivere gli aspetti fondamentali ed essenziali relativi all'attività di implementazione.\\

Inizieremo con una presentazione della struttura generale dei programmi assieme ad una descrizione accurata dei componenti utilizzati e delle scelte effettuate in fase di progettazione.\\

In seguito procederemo con un'attenta analisi degli algoritmi implementati, ovvero il l'algoritmo per il calcolo degli equilibri di Nash, quello per il calcolo dell'ottimo relativo alla funzione di benessere sociale utilitario e quello per il calcolo dell'ottimo relativo alla funzione di benessere sociale egalitario.\\
Tale sezione sarà caratterizzata dall'ultilizzo di pseudocodice per ciascuno degli algoritmi in modo tale da rendere più comprensibile la descrizione dei cicli e delle operazioni.\\

Tratteremo in modo approfondito questa porzione del documento poichè precede la sezione relativa alla sperimentazione effettuata attraverso i programmi implementati e dunque è di fondamentale importanza.\\

\section{Struttura generale}
\justify

Procediamo presentando la struttura generale relativa programmi implementati.\\ 
Quest'ultima è rappresentata attraverso una struttura ad albero che riproduce una porzione di filesystem partendo dalla root del progetto.\\
Al fine di rendere più chiara la lettura viene inoltre fornita un breve leggenda sulla nomenclatura utilizzata.\\

\begin{itemize}
	\item La nomenclatura \textbf{nome.dir} indica che l'oggetto è una cartella
	\item La nomenclatura \textbf{nome.edgelist} indica che l'oggetto è un file con estensione .edgelist (oggetto principale modellato dal programma)
	\item La nomenclatura \textbf{nome.dot} indica che l'oggetto è un file con estensione .dot (oggetto utilizzato su macchine GNU/Linux per il disegno attraverso la libreria Pygraphviz in fase di debug)
	\item le lettere \textbf{X,K,Y,Z,...} rappresentano numeri casuali (sono utilizzate per descrivere la moltitudine di cartelle, grafi creati e risultati, ottenuti durante un generale caso d'uso dei programmi)
	\item la nomenclatura \textbf{nome.init} indica che l'oggetto è un file con estensione .init (oggetto utilizzato in fase di lettura per salvare le caratteristiche dei grafi (nodi, archi, pesi, colorazione) e dei colori (colori, profitti))
	\item la nomenclatura \textbf{nome.out} indica che l'oggetto è un file con estensione .out (oggetto utilizzato in fase di lettura per salvare i risultati derivanti da esecuzioni singole o multiple usando gli algoritmi per il calcolo del nash, dell'ottimo con funzione di benessere sociale utilitario e dell'ottimo con funzione di benessere sociale egalitario)
\end{itemize}

Descriviamo ora la funzione basilare di alcuni componenti dell'albero sottostante rappresentante un esempio generale relativo ad un caso d'uso dei programmi.\\

\begin{itemize}
	\item La nomenclatura \textbf{nome.dir} indica che l'oggetto è una cartella
	
\end{itemize}

\newpage
\dirtree{%
.1 /.
.2 generator.dir.
.3 gen.dir.
.4 gen-dir-1.dir.
.5 graph-1.edgelist.
.5 graph-1.dot.
.4 .....
.4 gen-dir-X.dir.
.5 graph-X.edgelist.
.5 graph-X.dot.
.3 m-gen.dir.
.4 m-gen-dir-1.dir. 
.5 graph-1.edgelist.
.5 graph-1.dot.
.5 .....
.5 graph-K.edgelist.
.5 graph-K.dot.
.4 .....
.4 m-gen-dir-Y.dir.
.5 graph-1.edgelist.
.5 graph-1.dot.
.5 .....
.5 graph-Z.edgelist.
.5 graph-Z.dot.
.3 generator.py.
.2 reader.dir.
.3 result.dir.
.4 result-dir-1.dir.
.5 graph-1.init.
.5 graph-1.out.
.4 .....
.4 result-dir-X.dir.
.5 graph-X.init.
.5 graph-X.out.
.3 m-result.dir.
.4 m-result-dir-1.dir.
.5 graph-1.init.
.5 graph-1.out.
.4 .....
.4 m-result-dir-X.dir.
.5 graph-X.init.
.5 graph-X.out.
}