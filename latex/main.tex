\documentclass[11pt,a4paper,oneside,openany]{book}

%\usepackage{showframe}
\usepackage{mathtools}
\usepackage{blindtext}
\usepackage{multirow}
\usepackage{array}
\usepackage{amssymb}
\usepackage{dsfont}
\usepackage{dirtree}
\usepackage{algpseudocode}
\usepackage{float}

\usepackage[backend=bibtex]{biblatex}
\bibliography{bibl} 

\renewcommand{\arraystretch}{1.5}

%\usepackage{enumitem}

%\usepackage{titlesec}
%\titlespacing*{\chapter}{0pt}{-20pt}{20pt}
%\titleformat{\chapter}[display]{\normalfont\huge\bfseries}{\chaptertitlename\ \thechapter}{20pt}{\Huge}

\usepackage{graphicx}
\graphicspath{ {./img/} }

\usepackage{fancyhdr}
 
\pagestyle{fancy}
\fancyhf{}

\fancyhead[CE,CO]{\leftmark}
\fancyfoot[CE,CO]{\thepage}
 
\renewcommand{\headrulewidth}{0pt}
\renewcommand{\footrulewidth}{0pt}

\usepackage[utf8]{inputenc}
\usepackage[italian]{babel}
%\usepackage[T1]{fontenc}
\usepackage[document]{ragged2e}
%\renewcommand{\familydefault}{\rmdefault}

\usepackage{amsthm}
\newtheorem{prop}{Proposizione}
\newtheorem{theorem}{Teorema}

\begin{document}

    \documentclass[12pt,a4paper,oneside]{book}

\usepackage[utf8]{inputenc}
\usepackage[italian]{babel}

\usepackage{showframe}
%\usepackage{blindtext}

\usepackage{graphicx}
\graphicspath{ {./img/} }

%\usepackage{geometry}


\begin{document}


\begin{titlepage}

	\centering
	
	\includegraphics[height=5cm]{logowab}	
	
	\vspace{0.5cm}
	
	{\scshape\LARGE Università degli Studi dell'Aquila \par}

	\vspace{1cm}

	{\scshape\Large progetto di tesi \par}

	\vspace{1cm}

	{\huge\bfseries Calcolo e performance di equilibri di Nash per il gioco della k-colorazione generalizzata \par}

	\vspace{1.5cm}

	{\Large\itshape Valentino Di Giosaffatte \par}

	\vfill
	Relatore\par
	Dr.~Gianpiero \textsc{Monaco}

	\vfill

	% Bottom of the page

	{\large \today\par}
	
	\vspace{0.2cm}
	
\end{titlepage}


\chapter{evtbetryb}


\end{document}


    \tableofcontents
    
    \part{Equilibri di Nash}    
    \begin{flushleft}
\justify

\chapter{Introduzione agli equilibri di Nash}

\section{Teoria dei giochi}
La teoria dei giochi è la disciplina scientifica che si occupa dello studio e dell'analisi del comportamento e delle decisioni di soggetti razionali in un contesto di interdipendenza strategica.\\
Si definisce interdipendenza strategica, o interazione strategica, lo scenario in cui le decisioni di un individuo influenzano anche le scelte e gli scenari relativi agli altri individui.\\
Il principale oggetto di studio della teoria dei giochi sono le situazioni di conflitto nelle quali gli attori sono costretti ad intraprendere una strategia di competizione o cooperazione.\\
Tale scenario è definito gioco strategico e gli individui sono denominati giocatori.\\
Sulla base delle premesse e delle regole compositive del gioco in oggetto, viene costruito un modello matematico nel quale ciascun giocatore effettua le proprie decisioni (mosse migliorative) seguendo una strategia finalizzata ad aumentare il proprio vantaggio netto.\\
A ciascuna scelta positiva corrisponde un ritorno favorevole in termini di beneficio (payoff), il medesimo concetto vale in modo contrario in caso di scelta negativa, in tal caso il ritorno sarà sfavorevole.\\
In tali scenari le decisioni di un soggetto possono influire direttamente sui risultati conseguibili dagli altri e viceversa secondo un meccanismo di retroazione.\\
La teoria dei giochi è un concetto di soluzione applicabile ad un'ingente molteplicità di casi nei quali una pluralità di agenti decisionali possono operare in maniere competitiva, seguendo interessi contrastanti, o in maniera  cooperativa, seguendo l'interesse comune.\newline

\subsection{Giochi non-cooperativi}
In questo documento, la trattazione sarà incentrata sull'analisi di una particolare tipologia di giochi : i giochi non-cooperativi.\\
I giochi non-cooperativi, detti anche competitivi, rappresentano una specifica classe di giochi nella quale i giocatori non possono stipulare accordi vincolanti di cooperazione (anche normativamente), indipendentemente dai loro obiettivi.\\
Il criterio di comportamento razionale adottato nei giochi non-cooperativi è di carattere individuale ed è denominato strategia del massimo.\\
La suddetta definizione di razionalità va a modellare il comportamento di un individuo intelligente e ottimista che si prefigge l'obiettivo di prendere sempre la decisione che consegue il massimo guadagno possibile, perseguendo di conseguenza sempre la strategia più vantaggiosa per se stesso.\\
Si parla dunque di punto di equilibrio qualora nel gioco esista una strategia che presenti il massimo guadagno per tutti i giocatori, ovvero uno stato stabile del gioco nel quale tutti gli attori ottengono il massimo profitto individuale e collettivo.\newline

\section{Equilibri di Nash}
La precedente affermazione muove l'oggetto della trattazione verso l'argomento centrale di questo studio, ovvero gli equilibri di Nash.\\
L'equilibrio di Nash è una combinazione di strategie nella quale ciascun giocatore effettua la migliore scelta possibile, seguendo cioè una strategia dominante, sulla base delle aspettative di scelta dell'altro giocatore.\\
L'equilibrio di Nash è la combinazione di mosse (m1, m2) in cui la mossa di ciascun giocatore è la migliore risposta alla mossa effettuata da un altro giocatore.\\
Ciascun giocatore formula delle aspettative sulla scelta dell'altro giocatore e in base a queste decide la propria strategia, con l'obiettivo di massimizzare il proprio profitto e di conseguenza quello degli altri.
Un equilibrio di Nash è un equilibrio stabile, poiché nessun giocatore ha interesse a modificare la propria strategia.\\
Ciascun giocatore trae la massima utilità possibile dalle proprie scelte, tenendo conto della migliore scelta dell'altro giocatore, e dunque qualunque variazione alla propria strategia potrebbe soltanto peggiorare il proprio valore di tornaconto (payoff o utilità).\\
L'equilibrio di Nash è conosciuto anche con il nome di equilibrio non cooperativo poiché rappresenta una situazione di equilibrio ottimale per un gioco non-cooperativo.\\
L'equilibrio di Nash non deriva dall'accordo tra i giocatori, bensì dall'adozione di strategie dominanti perseguite da tutti i giocatori, tali da garantire sia il miglior profitto possibile per ciascun giocatore (ottimo individuale), sia il miglior equilibrio collettivo (ottimo sociale).\newline

\subsection{Definizione formale}
Definiamo ora alcuni concetti basilari e chiariamo alcuni aspetti matematici della teoria dei giochi al fine di delineare in modo più accurato il concetto di equilibrio di Nash.\\
Un gioco è caratterizzato da :
\begin{itemize}
	\item un insieme G di giocatori, o agenti, che indicheremo con \(i = 1,\ldots,N\)
	\item un insieme S di strategie, costituito da un insieme di M vettori \[S_{i}=\left(s_{{i,1}},s_{{i,2}},\ldots,s_{{i,j}},\ldots,s_{{i,M_{i}}}\right)\] ciascuno dei quali contiene l'insieme l'insieme delle strategie che il giocatore i-esimo ha a disposizione, cioè l'insieme delle azioni che esso può compiere.\\(indichiamo con \(s_i\) la strategia scelta dal giocatore \(i\))
	\item un insieme U di funzioni \[u_{i}=U_{i}\left(s_{1},s_{2},\ldots,s_{i},\ldots,s_{N}\right)\] che associano ad ogni giocatore \(i\) il guadagno (detto anche payoff) \(u_i\) derivante da una data combinazione di strategie (il guadagno di un giocatore in generale non dipende solo dalla propria strategia ma anche dalle strategie scelte dagli avversari)
\end{itemize}
Un equilibrio di Nash per un dato gioco è una combinazione di strategie (che indichiamo con l'apice \(e\))
\[s_{1}^{e},s_{2}^{e},...,s_{N}^{e}\]
tale che
\[U_{i}\left(s_{1}^{e},s_{2}^{e},...,s_{i}^{e},...,s_{N}^{e}\right)\geq U_{i}\left(s_{1}^{e},s_{2}^{e},...,s_{i},...,s_{N}^{e}\right)\]
\(\forall i\) e \(\forall s_i\) scelta dal giocatore i-esimo.\newline

Il significato di quest'ultima disuguaglianza è il seguente : se un gioco ammette almeno un equilibrio di Nash, ciascun agente ha a disposizione almeno una strategia \(s_i^e\) dalla quale non ha alcun interesse ad allontanarsi se tutti gli altri giocatori hanno giocato la propria strategia \(s_j^e\).\\
Come si può facilmente desumere direttamente dalla suddetta disequazione, se il giocatore i gioca una qualunque strategia a sua disposizione diversa da \(s_i^e\), mentre tutti gli altri giocatori hanno giocato la propria strategia \(s_j^e\), può solo peggiorare il proprio guadagno o, al più, lasciarlo invariato.\\
Da qui si può dedurre quindi che se i giocatori raggiungono un equilibrio di Nash, nessuno può più migliorare il proprio risultato modificando solo la propria strategia, ed è quindi vincolato alle scelte degli altri.\\
Poiché questo vale per tutti i giocatori, è evidente che se esiste un equilibrio di Nash ed è unico, esso rappresenta la soluzione del gioco, in quanto nessuno dei giocatori ha interesse a cambiare strategia.\newline

\section{Nozioni di base}

\subsection{Rappresentazione con matrici di payoff e descrizione del procedimento decisionale}
Nella seguente matrice di payoff è rappresentato un esempio di equilibrio di Nash in un gioco non-cooperativo a 2 giocatori (tale equilibrio può essere esteso a N giocatori).

\begin{center}
  \begin{tabular}{ | c | c | c | }
    \cline{2-3}
    \multicolumn{1}{ c| }{} & \multicolumn{2}{ c| }{\textbf{giocatore 2}} \\ \cline{2-3}
    \multicolumn{1}{ c| }{} & \textbf{S1} & \textbf{S1} \\ \hline
    \textbf{S1} & A [2;2] & B [2;1] \\ \hline
    \textbf{S2} & C [1;2] & D [1;1] \\
    \hline
  \end{tabular}
\end{center}

Nella suddetta matrice ciascun giocatore può scegliere S1 o la strategia S2. Il giocatore 1 si aspetta che il giocatore 2 scelga S1 (strategia dominante) e, quindi, adotta anch'egli la strategia S1 in quanto gli consente di ottenere un payoff individuale pari a 2.\\
Anche il giocatore 2 formula delle aspettative sulle scelte dell'avversario e si attende che il giocatore 1 scelga S1, scegliendo a sua volta la strategia S1.\\
L'equilibrio del gioco converge verso la cella A della matrice nella quale entrambi i giocatori massimizzano il proprio payoff individuale (ottimo individuale) dopo aver scelto la propria strategia dominante. Entrambi i giocatori hanno formulato un'ipotesi sulla migliore scelta del giocatore avversario e, sulla base di questa, hanno scelto la propria strategia dominante.\newline

Cerchiamo ora di comprendere al meglio il funzionamento del processo decisionale alla base dell'equilibrio di Nash.\\
Il giocatore 1 avrebbe potuto scegliere sia S1 che S2, in entrambi i casi avrebbe potuto sperare di ottenere per sé il payoff più alto (2) nelle celle A e D. Il giocatore 1 sa bene però che se scegliesse S2, il giocatore 2 sceglierebbe in seguito S1 e l'equilibrio finale si collocherebbe nella cella C, nella quale egli otterrebbe il payoff più basso (1).\\
Qualora scegliesse invece S1, il giocatore 1 sarebbe consapevole che anche il giocatore 2 sceglierebbe S1 e l'equilibrio finale in questo caso si collocherebbe nella cella A, nella quale il giocatore 1 otterrebbe il payoff più alto (2).\\
Seguendo il medesimo ragionamento, qualora spettasse al giocatore 2 scegliere per primo, questi sarebbe consapevole che scegliendo S1 anche il giocatore 1 sceglierebbe S1. Dunque anche il questo caso l'equilibrio strategico si collocherebbe nella cella A.\\
Il giocatore 2 non sceglierebbe mai S2 in quanto in ogni caso avrebbe il payoff più basso (1).\newline

\subsection{Equilibri di Nash e ottimo sociale}
Nell'esempio appena trattata possiamo inoltre constatare facilmente che l'equilibrio di Nash trovato (cella A della matrice) è anche un ottimo sociale. Nel suddetto equilibrio coesistono una situazione ottimale per entrambi i giocatori (entrambi i giocatori possiedono un payoff massimo) e una situazione ottimale per l'intera collettività (in quanto la somma dei valori di payoff di entrambi i giocatori è uguale 4, il valore maggiore all'interno della matrice)\\
Specifichiamo inoltre che un gioco non-cooperativo può presentare più equilibri di Nash. Anche il presenza di equilibri multipli, ciascun equilibrio è comunque stabile, poiché dalla propria posizione di equilibrio locale qualsiasi scelta è peggiorativa per ogni giocatore.\\
Al fine di verificare quest'ultima affermazione viene presentata la seguente matrice di payoff nella quale sono presenti 2 equilibri di Nash simmetrici (nelle celle A e D).\\

\begin{center}
  \begin{tabular}{ | c | c | c | }
    \cline{2-3}
    \multicolumn{1}{ c| }{} & \multicolumn{2}{ c| }{\textbf{giocatore 2}} \\ \cline{2-3}
    \multicolumn{1}{ c| }{} & \textbf{S1} & \textbf{S1} \\ \hline
    \textbf{S1} & A [2;2] & B [2;1] \\ \hline
    \textbf{S2} & C [1;2] & D [2;2] \\
    \hline
  \end{tabular}
\end{center}

\subsection{Assenza di equilibri di Nash}
In alcuni casi possono essere del tutto assenti le condizioni per determinare un equilibrio di Nash e di conseguenza molti giochi sono privi di equilibrio. Per verificare tale asserzione mostriamo la matrice di payoff relativa ad un gioco nel quale non è possibile trovare un equilibrio di Nash.\\

\begin{center}
  \begin{tabular}{ | c | c | c | }
    \cline{2-3}
    \multicolumn{1}{ c| }{} & \multicolumn{2}{ c| }{\textbf{giocatore 2}} \\ \cline{2-3}
    \multicolumn{1}{ c| }{} & \textbf{S1} & \textbf{S1} \\ \hline
    \textbf{S1} & A [2;2] & B [2;1] \\ \hline
    \textbf{S2} & C [1;2] & D [2;2] \\
    \hline
  \end{tabular}
\end{center}

Nell'esempio appena descritto, se il giocatore 1 sceglie la strategia S2 (tentando di ottenere il payoff 2), il giocatore 2 sceglierà la strategia S2 (payoff 2) e l'equilibrio si posiziona nella cella D della matrice.\\
Se al contrario è il giocatore 2 a scegliere la strategia S2 (tentando a sua volta di ottenere il payoff 2), il giocatore 1 sceglierà la strategia S1 (payoff 1) e l'equilibrio si posiziona nella cella B della matrice. Nel suddetto esempio non esiste dunque un equilibrio di Nash.\newline

\subsection{Il dilemma del prigioniero : sub-ottimalità individuale e sociale}
L'adozione di strategie dominanti non assicura sempre un equilibrio di ottimo sociale. In assenza di informazioni, un gioco non-cooperativo potrebbe convergere verso un equilibrio stabile ma non ottimale.\\
L'ipotesi è dimostrata nel problema del "dilemma del prigioniero" in cui due attori, pur formulando delle aspettative razionali sul comportamento dell'avversario e adottando strategie dominanti, determinano un equilibrio sub-ottimale ( D ) sia dal punto individuale che sociale.\\

\begin{center}
  \begin{tabular}{ | c | c | c | }
    \cline{2-3}
    \multicolumn{1}{ c| }{} & \multicolumn{2}{ c| }{\textbf{giocatore 2}} \\ \cline{2-3}
    \multicolumn{1}{ c| }{} & \textbf{non confessa} & \textbf{confessa} \\ \hline
    \textbf{non confessa} & A [5;5] & B [0;9] \\ \hline
    \textbf{confessa} & C [9;0] & D [2;2] \\
    \hline
  \end{tabular}
\end{center}

Nell'esempio il giocatore 1 si aspetta che il giocatore 2 confessi, poiché la strategia dominante del giocatore 2 è confessare poiché grazie alla confessione quest'ultimo otterrebbe la libertà (payoff 9).\\
Sulla base di questa aspettativa il giocatore 1 sceglie la propria strategia dominante tra B e D, scegliendo a sua volta di confessare per ottenere un payoff uguale a 2. Lo stesso ragionamento viene adottato, in modo inverso, dal giocatore 2.\\
L'equilibrio D è un equilibrio di Nash ed è un equilibrio stabile poiché nessuno dei 2 giocatori ha interessa a modificare le proprie scelte.
In tale equilibrio però entrambi i giocatori ottengono un payoff sub-ottimale pari a 2 rispetto a quello che avrebbero se decidessero entrambi di non confessare (payoff 5 - equilibrio A) e in più l'equilibrio D presenta un valore di ottimo sociale (4) inferiore rispetto a quello che si può ottenere dall'equilibrio A (10) che rappresenta l'ottimo individuale e sociale del gioco.

\end{flushleft}
    
    \part{Gioco della $k$-colorazione generalizzata}
    \chapter{Gioco della k-colorazione generalizzata}
\justify
Esaminiamo ora gli equilibri di Nash puri per il gioco della k-colorazione generalizzata nel quale viene fornito un grafo orientato e un insieme di k colori.\\
I nodi rappresentano i giocatori e gli archi catturano i loro reciproci interessi.\\
La strategia di ciascun giocatore è composta da k colori.\\
L'utilità di un giocatore v in un dato stato o colorazione è data dalla somma dei pesi degli archi \((v, u)\) incidenti a v tale che il colore scelto da v sia diverso da quello scelto da u, più il profitto guadagnato dall'utilizzo del colore scelto.\\
Per prima cosa dimostriamo che il gioco della k-colorazione generalizzata è convergente e dunque esiste sempre almeno un equilibrio di Nash per ogni istanza del gioco in questione.\\
Valutiamo dunque in seguito una descrizione delle prestazioni dei giochi della k-colorazione generalizzata per mezzo delle nozioni largamente utilizzate di prezzo dell'anarchia (price of anarchy) e prezzo della stabilità (price of stability).\\
Forniamo inoltre limiti stretti per 2 tipi di benessere sociale ampiamente utilizzati, il benessere sociale utilitario (utilitarian social welfare) e il benessere sociale egalitario (egalitarian social welfare).\newline

\section{Descrizione generale}
\justify
Le istanze appartenenti al gioco della k-colorazione generalizzata sono giocati su grafi non-orientati pesati in cui i nodi corrispondono ai giocati e in cui gli archi identificano le connessioni sociali o le relazioni tra i giocatori.\\
Il set di strategie di ciascun giocatore è un insieme di k colori disponibili (assumiamo che i colori siano gli stessi per ogni giocatore).\\
Quando i giocatori selezionano un colore inducono una colorazione k (o semplicemente una colorazione).\\
Ciascun giocatore possiede una funzione di profitto legata all'apprezzamento da parte di quest'ultimo del colore scelto (vale per tutti i colori disponibili per il giocatore).\\
Data una colorazione, l'utilità (o il guadagno) di un giocatore v colorato con il colore i è la somma dei pesi degli archi \((v, u)\) incidenti a v, tale che il colore scelto da v è diverso da quello scelto da u, più il profitto derivante dalla scelta del colore i da parte del giocatore v.\\
Assumiamo che i giocatori siano egoisti, dunque un concetto di soluzione ben noto per questo tipo di impostazione è l'equilibrio di Nash.\\
L'equilibrio di Nash è uno dei concetti più importanti nella teoria dei giochi e fornisce una soluzione stabile che è robusta alle deviazioni dei singoli giocatori.\\
Formalmente, una colorazione è un equilibrio di Nash puro se nessun giocatore può migliorare la propria utilità deviando unilateralmente dalla propria strategia attuale.\\
L'egoismo dei giocatori può causare in molti casi la perdita di benessere sociale e quindi una soluzione stabile non è sempre buona rispetto al benessere della società.\newline

Consideriamo ora 2 nozioni di benessere sociale, naturali e ampiamente utilizzate.\\
Data una colorazione k, il benessere sociale utilitario (utilitarian social welfare) è definito come la somma delle utilità dei giocatori nella colorazione k, mentre il benessere sociale egalitario (egalitarian sociale welfare) è definito come l'utilità minima tra tutti i giocatori nella colorazione k.\newline

Utilizziamo inoltre 2 metodi per misurare la bontà di un equilibrio di Nash rispetto a un benessere sociale, il prezzo dell'anarchia (price of anarchy) e il prezzo della stabilità (price of stability).\\
Il prezzo dell'anarchia descrive, nel peggiore dei casi, come l'efficienza di un sistema degrada a causa del comportamento egoistico dei suoi giocatori, mentre il prezzo della stabilità ha un naturale significato di stabilità, poiché è la soluzione ottimale tra quelle che possono essere accettate da giocatori egoisti.\\
Studiamo ora l'esistenza e le performance degli equilibri di Nash nei giochi della k-colorazione generalizzata.\\
Ci concentriamo solo sui grafi non-orientati poiché per i grafi orientati anche il problema di decidere se un'istanza ammetta un equilibrio di Nash è un problema difficile (NP-Hard), inoltre esistono casi per i quali un equilibrio di Nash non esiste affatto.\newline

\subsection{Nozioni sul problema}
\justify
Sappiamo che in caso di grafi non-orientati non-pesati è possibile calcolare un equilibrio di Nash in tempo polinomiale.\\
Nel nostro caso, il problema di calcolare un equilibrio di Nash su grafi non-orientati pesati è PLS-Completo anche per \(k = 2\), dato che il gioco del taglio massimo (Max-cut game) è un caso speciale del nostro gioco.\\
Proprio riguardo questo aspetto è bene delineare la relazione che esiste tra il gioco della k-colorazione generalizzata e il gioco del taglio massimo, un problema molto importante e ampiamente trattato in letteratura.\\
Il gioco della k-colorazione generalizzata è un estensione del gioco del taglio massimo, infatti quest'ultimo può essere ottenuto ponendo a 0 i profitti relativi ai colori e ponendo a 2 il numero di colori presenti nel set disponibile per ciascun giocatore.\\
Inoltre il gioco della k-colorazione generalizzata è un'estensione del gioco della k-colorazione nel quale vi sono k-colori ma i profitti relativi ai colori sono impostati a 0.\newline

\section{Dettagli sul modello}
\justify
Dato un grafo semplice non-orientato \(G = (V, E, w)\), dove \(|V| = n\), \(|E| = m\) e \(w : E\rightarrow\mathds{R}_{\geq 0}\) è la funzione per i pesi sugli archi che associa un peso positivo a ciascun arco.\\
Denotiamo con \(\delta^v (G) = \sum_{u \in V : \{v, u\} \in E} w(\{v, u\})\) la somma dei pesi di tutti gli archi incidenti a v.\\
L'insieme dei nodi con cui un nodo v ha un arco in comune è chiamato insieme dei vicini di v (insieme dei nodi adiacenti a v).\\
Un'istanza di gioco della k-colorazione generalizzata è un tupla \((G, K, P)\). \(G = (V, E, w)\) è un grafo pesato non-orientato senza self loops, in cui ogni \(v \in V\) è un giocatore egoista.\\
K è un insieme di colori disponibili (assumiamo \(K \geq 0\)).\\
Il set di strategia di ciascun giocatore è dato dai k colori disponibili, ovvero i giocatori hanno lo stesso insieme di azioni.\\
Denotiamo con \(P : V \times K \rightarrow \mathds{R}_{\geq 0}\) la funzione di profitto del colore, che definisce quanto un giocatore apprezza un colore, ovvero se il giocatore v scegli di usare il colore i, allora guadagna \(P_v (i)\).\\
Per ciascun giocatore v, definiamo \(P_v^M\) come il massimo profitto che v può guadagnare da un colore, formalmente \(P_v^M = max_{i=1,\ldots,k} P_v (i)\).\\
Quando \(P_v (i) = 0 \forall v \in V\) e \(\forall i \in k\), si ha il caso in cui non vi sono profitti associati ai colori scelti, quindi possiamo riferirci a questo gioco come a un gioco della k-colorazione (graph k-coloring game).\\
Uno stato del gioco \(c = \{c_1,\ldots,c_n\}\) è una k-colorazione, o semplicemente una colorazione, dove \(c_v\) è il colore (cioè un numero \(1 \leq c_v \leq k\)) scelto dal giocatore v.\\
In una determinata colorazione c, il payoff (o l'utilità) di un giocatore v è la somma dei pesi degli archi \((v, u)\) incidenti a v, tale che il colore scelto da v è diverso da quello scelto da u, oltre al profitto ottenuto dall'utilizzo il colore scelto.\\
In modo formale, per una colorazione c, il payoff di un giocatore v è \(\mu_c (v) = \sum_{u \in V:\{v, u\} \in E \wedge c_v \neq c_u} w(\{v, u\}) + P_v(c_v)\).\\
Quando un arco \((v, u)\) fornisce utilità ai suoi endpoints in una colorazione c, cioè quando \(c_v \neq c_u\) diciamo che tale arco è corretto.\\
diciamo anche che un arco \((v, u)\) è monocromatico in una colorazione c quanto \(c_v = c_u\).\\
Sia \(c_{-v}, c_u^{\prime}\) la colorazione ottenuta da c cambiando la strategia del giocatore v da \(c_v\) a \(c_v^{\prime}\).\\
Data una colorazione \(c = \{c_1,\ldots,c_n\}\), una mossa migliorativa (improving move) del giocatore v nella colorazione c è una strategia \(c_v^{\prime}\) tale che \(\mu_{(c_{-v}, c_v^{\prime})} (v) > \mu_c (v)\).\\
Uno stato del gioco è un equilibrio di Nash puro o equilibrio stabile se e so se nessun giocatore può effettuare una mossa migliorativa.\\
In modo formale, \(c = \{c_1,\ldots,c_n\}\) è un equilibrio di Nash se \(\mu_c (v) \geq \mu_{(c_{-v}, c_v^{\prime})} (v)\) per ogni possibile colorazione \(c_v^{\prime}\) e per ogni giocatore \(v \in V\).\\
Una dinamica di miglioramento (improving dynamic), o brevemente dinamica (dynamic), è una sequenza di mosse migliorative. Si dice che un gioco sia convergente se, dato un qualsiasi stato iniziale c, qualsiasi sequenza di mosse migliorative porta a un equilibrio di Nash.\newline
Data una colorazione c, definiamo una funzione di benessere sociale utilitario (utilitarian social welfare) \(SW_{UT}(c)\) e una funzione di benessere sociale egalitario (egalitarian sociale welfare) \(SW_{EG}(c)\) come segue :
\[SW_{UT} (c) = \sum_{v \in V} \mu_c (v) = \sum_{v \in V} P_v(c_v) + \sum_{\{v, u\} \in E : c_v \neq c_u} 2w(\{v, u\})\]
\[SW_{EG} (c) = min_{v \in V} \mu_c (v)\]
Indichiamo con C l'insieme di tutte le possibili colorazioni e denotiamo con Q l'insieme di tutte le colorazioni stabili. Data una funzione di benessere sociale SW, definiamo il prezzo dell'anarchia (price of anarchy) (PoA) per il gioco della k-colorazione generalizzata come il rapporto tra il massimo benessere sociale tra tutte le possibili colorazioni sul minimo benessere sociale tra tutte le possibili colorazioni stabili.\\
In modo formale, \(PoA = \frac{max_{c \in C} SW(c)}{min_{c^{\prime} \in Q} SW(c^{\prime})}\).\\
Definiamo inoltre il prezzo della stabilità (price of stability) (PoS) per il gioco della k-colorazione generalizzata come il rapporto tra il massimo benessere sociale tra tutte le possibili colorazioni sul massimo benessere sociale tra tutte le possibili colorazioni stabili.\\
In modo formale, \(PoS = \frac{max_{c \in C} SW(c)}{max_{c^{\prime} \in Q} SW(c^{\prime})}\).\\
Intuitivamente, il PoA (rispettivamente PoS) ci dice quanto è peggiore il benessere sociale nel peggiore (rispettivamente migliore) equilibrio di Nash, relativo al benessere sociale dell'ottimo.\\

\subsection{Convergenza ed esistenza degli equilibri di Nash}
\justify
    
    \part{Implementazione}
    \chapter{Implementazione}

In questo capitolo andremo a delineare e descrivere gli aspetti fondamentali ed essenziali relativi all'attività di implementazione.\\

Inizieremo con una presentazione della struttura generale dei programmi assieme ad una descrizione accurata dei componenti utilizzati e delle scelte effettuate in fase di progettazione.\\

In seguito procederemo con un'attenta analisi degli algoritmi implementati, ovvero il l'algoritmo per il calcolo degli equilibri di Nash, quello per il calcolo dell'ottimo relativo alla funzione di benessere sociale utilitario e quello per il calcolo dell'ottimo relativo alla funzione di benessere sociale egalitario.\\
Tale sezione sarà caratterizzata dall'utilizzo di pseudocodice per ciascuno degli algoritmi in modo tale da rendere più comprensibile la descrizione dei cicli e delle operazioni.\\

Tratteremo in modo approfondito questa porzione del documento poiché precede la sezione relativa alla sperimentazione effettuata attraverso i programmi implementati e dunque è di fondamentale importanza.\\

\section{Struttura generale}
\justify

Procediamo presentando la struttura generale relativa programmi implementati.\\ 
Quest'ultima è rappresentata attraverso una struttura ad albero che riproduce una porzione di filesystem partendo dalla root del progetto.\\
Al fine di rendere più chiara la lettura viene inoltre fornita un breve leggenda sulla nomenclatura utilizzata.

\begin{itemize}
	\item La nomenclatura \textbf{nome.dir} indica che l'oggetto è una cartella
	\item La nomenclatura \textbf{nome.edgelist} indica che l'oggetto è un file con estensione .edgelist (oggetto principale modellato dal programma)
	\item La nomenclatura \textbf{nome.dot} indica che l'oggetto è un file con estensione .dot (oggetto utilizzato su macchine GNU/Linux per il disegno attraverso la libreria PyGraphViz in fase di debug)
	\item le lettere \textbf{X,K,Y,Z,H,W} rappresentano numeri casuali (sono utilizzate per descrivere la moltitudine di cartelle, grafi creati e risultati, ottenuti durante un generale caso d'uso dei programmi)
	\item la nomenclatura \textbf{nome.init} indica che l'oggetto è un file con estensione .init (oggetto utilizzato in fase di lettura per salvare le caratteristiche dei grafi (nodi, archi, pesi, colorazione) e dei colori (colori, profitti))
	\item la nomenclatura \textbf{nome.out} indica che l'oggetto è un file con estensione .out (oggetto utilizzato in fase di lettura per salvare i risultati derivanti da esecuzioni singole o multiple usando gli algoritmi per il calcolo del nash, dell'ottimo con funzione di benessere sociale utilitario e dell'ottimo con funzione di benessere sociale egalitario)
\end{itemize}

Descriviamo ora la funzione basilare di alcuni componenti dell'albero sottostante rappresentante un esempio generale relativo ad un caso d'uso dei programmi.

\begin{itemize}
	\item La cartella \textbf{generator} contiene al suo interno l'intera struttura relativa al generatore di grafi
	\item La cartella \textbf{gen} contiene i risultati delle generazioni singole di grafi
	\begin{enumerate}
		\item Al suo interno vi sono X cartelle gen-dir-X (il nome viene assegnato dinamicamente dall'utente) generate dinamicamente
		\item Ciascuna cartella contiene il risultato di una generazione singola di un grafo sotto forma di una coppia di file .edgelist e .dot (utilizzato solo in fase di debug)
		\item Ogni gruppo cartella-file.edgelist-file.dot rappresenta il risultato di una generazione singola
	\end{enumerate}
	\item La cartella \textbf{m-gen} contiene i risultati delle generazioni multiple di grafi
	\begin{enumerate}
		\item Al suo interno vi sono Y cartelle m-gen-dir-Y (il nome viene assegnato dinamicamente dall'utente) generate dinamicamente
		\item Ciascuna cartella contiene il risultato di una generazione multipla di grafi sotto forma di molteplici coppie (nell'esempio K,Z,...) di file .edgelist e .dot (utilizzato solo in fase di debug)
		\item Ogni gruppo cartella-file.edgelist-file.dot-file.edgelist-file.dot-... indica il risultato di una generazione multipla
	\end{enumerate}
	\item Il file \textbf{generator.py} contiene al suo interno il codice del generatore di grafi scritto interamente in linguaggio Python
\end{itemize}

\begin{itemize}
	\item La cartella \textbf{reader} contiene al suo interno l'intera struttura relativa al lettore di grafi
	\item La cartella \textbf{result} contiene i risultati delle letture / sperimentazioni su grafi singoli
	\begin{enumerate}
		\item Al suo interno vi sono H cartelle result-dir-H (il nome viene preso dinamicamente dal grafo in lettura) generate dinamicamente
		\item Ciascuna cartella contiene il singolo risultato di una lettura / sperimentazione su un grafo sotto forma di una coppia di file .init e .out
		\item Ogni gruppo cartella-file.init-file.out rappresenta il risultato di una singola lettura / sperimentazione 
	\end{enumerate}
	\item La cartella \textbf{m-result} contiene i risultati delle letture / sperimentazioni su molteplici grafi
	\begin{enumerate}
		\item Al suo interno vi sono W cartelle m-result-dir-W (il nome viene preso dinamicamente dalla cartella relativa alla moltitudine di grafi in lettura) generate dinamicamente
		\item Ciascuna cartella contiene multipli risultati di multiple letture / sperimentazioni su grafo multipli sotto forma di una coppia di file .init e .out
		\item Ogni gruppo cartella-file.init-file.out rappresenta il risultato di una lettura / sperimentazione multipla
	\end{enumerate}
	\item Il file \textbf{reader.py} contiene al suo interno il codice del lettore di grafi scritto interamente in linguaggio Python 
\end{itemize}

\newpage
\dirtree{%
.1 /.
.2 generator.dir.
.3 gen.dir.
.4 gen-dir-1.dir.
.5 graph-1.edgelist.
.5 graph-1.dot.
.4 .....
.4 gen-dir-X.dir.
.5 graph-X.edgelist.
.5 graph-X.dot.
.3 m-gen.dir.
.4 m-gen-dir-1.dir. 
.5 graph-1.edgelist.
.5 graph-1.dot.
.5 .....
.5 graph-K.edgelist.
.5 graph-K.dot.
.4 .....
.4 m-gen-dir-Y.dir.
.5 graph-1.edgelist.
.5 graph-1.dot.
.5 .....
.5 graph-Z.edgelist.
.5 graph-Z.dot.
.3 generator.py.
.2 reader.dir.
.3 result.dir.
.4 result-dir-1.dir.
.5 graph-1.init.
.5 graph-1.out.
.4 .....
.4 result-dir-H.dir.
.5 graph-H.init.
.5 graph-H.out.
.3 m-result.dir.
.4 m-result-dir-1.dir.
.5 graph-1.init.
.5 graph-1.out.
.4 .....
.4 m-result-dir-W.dir.
.5 graph-W.init.
.5 graph-W.out.
}

\section{Componenti utilizzati e progettazione}
\justify
I programmi generator.py e reader.py sono stati interamente scritti utilizzando il linguaggio Python [versione 3.6.5].\\

Il codice è stato scritto utilizzando differenti editor di testo (vim [neovim], spacemacs, sublime-text, atom,...) e testato su diverse macchine GNU/Linux e Windows, in particolare su differenti shell (bash, zsh, fish) e su cmd.\\

Per agevolare il processo di progettazione e scrittura del codice l'intera struttura del progetto è stata caricata in una repository sul sito github e gestita in remoto (attraverso il software git).\\

Per questioni di compatibilità e versatilità è stata utilizzata in modo massiccio la libreria standard relativa al linguaggio Python per effettuare la quasi totalità delle operazioni in entrambi i programmi.\\

La versione di riferimento della libreria è quella associata alla versione del linguaggio utilizzato, dunque la 3.6.5.\\

Al fine di trascurare alcuni aspetti relativi alla strutturazione e costruzione dei grafi è stata utilizzata una potente libreria per la creazione e la manipolazione di questi oggetti matematici, ovvero NetworkX.\\

Tale scelta di progettazione ha permesso al sottoscritto di concentrarsi maggiormente sulla progettazione e sull'ottimizzazione degli algoritmi.\\

Inoltre tale scelta ha consentito al sottoscritto di rendere totalmente dinamici i processi di creazione e lettura dei grafi, ciò ha facilitato di molto il carico di lavoro in fase di sperimentazione.\\

La libreria standard del linguaggio Python è stata utilizzata in particolare per rendere totalmente dinamica e cross-platform la gestione del filesystem.\\

Ciò è stato necessario per garantire il funzionamento asincrono del generatore e del lettore, in modo tale da facilitare il lavoro in fase di sperimentazione.\\

Le fasi di generazione e lettura dei grafi infatti sono state completamente separate al livello di utilizzo, per fare ciò è stato necessario manipolare efficientemente il filesystem in modo da salvare i grafi creati e i risultati delle sperimentazioni su file.\\

Tali operazioni sono perfettamente funzionanti sia su sistemi che rispettano lo standard POSIX (Sistemi Unix-like) per il filesystem sia per sistemi che non lo rispettano (Sistemi Windows), dunque l'intero progetto è totalmente cross-platform e può essere facilmente migrato rendendo la portabilità un importante fattore di forza di quest'ultimo.\\

Il formato principale manipolato dai programmi è il formato .edgelist che analizzeremo in seguito.\\

In fase di debug è stata utilizzata la libreria di disegno MatPlotLib per rappresentare i grafi, quest'ultima è integrata in modo nativo all'interno della libreria NetworkX, dunque tale scelta di utilizzo ha reso più facili le operazioni di analisi e debug.\\

Un'altra libreria che è stata utilizzata in fase di analisi e debug per rappresentare i grafi in ambiente GNU/Linux è la libreria di disegno PyGraphViz.\\

Anche quest'ultima è integrata in modo nativo all'interno della libreria NetworkX.\\

Il formato di descrizione testuale dei grafi .dot è stato utilizzato solo in fase di debug in ambiente GNU/Linux assieme alla libreria di disegno PyGraphViz, dunque possiamo tralasciare la sua definizione e descrizione poiché non viene utilizzato all'interno dei programmi durante l'esecuzione.\\

Altre librerie minori sono state scelte in fase di progettazione ed utilizzate all'interno dei programmi, ad esempio la libreria per la colorazione dell'output testuale su terminal emulators cross-platform Colorama è stata utilizzata in fase di analisi e debug per semplificare e rendere più chiara la lettura dell'output relativo all'esecuzione degli algoritmi.\\

Un altro esempio è l'utilizzo della libreria cross-platform Pick, che rende semplice ed efficace la selezione delle opzioni all'interno dei terminal emulators durante l'esecuzione dei programmi.\\

L'elenco completo delle librerie utilizzate all'interno dei programmi è il seguente.

\begin{itemize}
	\item Python Standard Library (Python 3.6.5 - Python 2.7.14)\\
	https://docs.python.org/3/library/
	\item NetworkX Library (NetworkX 2.1)\\ 	
	https://pypi.org/project/networkx/	
	\item Matplotlib Library (Matplotlib 2.2)\\
	https://pypi.org/project/matplotlib/
	\item Pick Library (Pick 0.6.4)\\
	https://pypi.org/project/pick/
	\item Pydot Library (Pydot 1.2.4)\\
	https://pypi.org/project/pydot/
	\item Graphviz Library (Graphviz 0.8.2)\\
	https://pypi.org/project/graphviz/
	\item PyParsing Library (PyParsing 2.2.0)\\
	https://pypi.org/project/pyparsing/
	\item Colorama Library (Colorama 0.3.9)\\
	https://pypi.org/project/colorama/
\end{itemize}

Per il funzionamento completo dei programmi è necessaria l'installazione di Python e dei suddetti componenti attraverso l'uso del modulo pip e/o l'uso di package manager (apt, pacman, ...).\\

Procediamo ora nella trattazione descrivendo il funzionamento dei programmi generator.py e reader.py.\\
Saranno poi analizzati in modo approfondito gli algoritmi per il calcolo degli equilibri di Nash e degli ottimi.\\

\section{generator.py : il generatore di grafi}
\justify
Il programma generator.py è un generatore dinamico di grafi scritto interamente in Python in grado di semplificare le operazioni di creazione e manipolazione di questi oggetti matematici e in grado di creare automaticamente tutte le strutture di filesystem necessarie al salvataggio dei grafi generati su file.\\

Il programma come specificato in precedenza è completamente cross-platform e per ciò che riguarda la gestione del filesystem è stato accuratamente ottimizzato per non generare conflitti e problemi di inconsistenza dei dati.\\

Principalmente il programma utilizza due moduli principali necessari al corretto funzionamento dello stesso, la Standard Library del linguaggio Python per le funzioni di base e la libreria NetworkX per la creazione e manipolazione dei grafi.\\

Di seguito vengono riportate le caratteristiche del programma assieme ad un esempio generale di un caso d'uso.\\

Per prima cosa vengono impostati i motori di disegno, ovvero le librerie MatPlotLib e PyGraphViz.\\
Quest'ultima è stata utilizzata solo in ambiente GNU/Linux in fase di debug, dunque non è disponibile per l'utente.\\

In compenso è però disponibile per l'utente la libreria MatPlotLib che consente a quest'ultimo, qualora volesse, di disegnare al termine della generazione i grafi appena creati.\\
È di fondamentale importanza però specificare che il processo di disegno per grafi di grandi dimensioni è molto dispendioso e dunque può richiedere un tempo considerevole.\\

Come prima operazione all'avvio del programma è disponibile per l'utente una scelta della classe di grafo da generare effettuata attraverso un interfaccia di selezione da console implementata grazie alla libreria Pick.\\

Sono disponibili per l'utente 2 modalità di funzionamento per la generazione, la modalità SINGLE MODE (la modalità di default del programma) e la modalità MULTIPLE MODE (accessibile attraverso la selezione dell'opzione MULTIPLE nella scelta della classe).\\

\subsection{generator.py : SINGLE MODE}
\justify
Le classi implementate fanno riferimento a quelle presenti nella libreria NetworkX, ovvero le seguenti (se ne citano solo alcune).

\begin{itemize}
	\item Classic
	\item Expanders
	\item Small 
	\item Random graphs 
	\item Duplication divergence 
	\item ...
	\item MULTIPLE
\end{itemize}

Una volta selezionata la classe desiderata l'utente si troverà davanti una nuova interfaccia di selezione da console implementata attraverso l'uso del modulo Pick.\\

L'opzione MULTIPLE cambia la tipologia di funzionamento del programma, che da SINGLE MODE (modalità di generazione di un singolo grafo per volta) passa a MULTIPLE MODE (nella quale possono essere generati da 1 a n grafi della stessa tipologia in un solo processo di creazione, con n scelto dall'utente).\\
Concentriamoci ora sulla SINGLE MODE, la default mode del programma.\\

Questa volta l'utente dovrà scegliere la tipologia di grafo da creare.\\
Per ogni classe vi sono molteplici tipologie di grafo che l'utente può scegliere di generare.\\

Le tipologie implementate fanno riferimento a quelle presenti nel modulo NetworkX, in particolare ciascuna tipologia ha un costruttore di libreria corrispondente che provvede a generare le strutture elementari e a comporre l'oggetto matematico richiesto dall'utente.\\

Solo per far comprendere al lettore la vastità delle scelte possibili per l'utente in fase di selezione del grafo da generare, vengono qui si seguito riportate le tipologie di grafi implementate per la sola classe di grafi Classic.

\begin{itemize}
	\item balanced tree
	\item complete graph
	\item circular ladder graph
	\item cycle graph
	\item dorogovtsev goltsev mendes graph
	\item ladder graph
	\item lollipop graph
	\item path graph
	\item star graph
	\item turan graph
	\item wheel graph
	\item ...
\end{itemize}

Una volta selezionata la tipologia di grafo da implementare, l'utente potrà inserire i parametri di creazione per quest'ultimo.\\
I parametri associati ai grafi variano da tipologia a tipologia e sono necessari e fondamentali per la corretta generazione del grafo scelto.\\

Per una migliore comprensione, ad esempio i parametri associati alla tipologia balanced tree sono il branching factor e l'height dell'albero, invece per la tipologia di grafo complete graph vi è un unico parametro da passare al programma, il node number del grafo.\\
L'esempio ovviamente ricopre solo 2 tipologie di grafo, ma ciò vale per ogni tipologia implementata nel programma.\\

A questo punto l'utente è chiamato a scegliere quale tipologia di dato utilizzare per codificare i pesi associati agli archi del grafo, peso flottante (tipo float) o peso intero (tipo int).\\
Possiamo tralasciare il tipo flottante, dato che quest'ultimo, non essendo interessante e significativo per la sperimentazione, è stato tralasciato in favore del tipo intero.\\

L'utente dunque dovrà scegliere i valori massimo e minimo relativi al range all'interno del quale oscilleranno randomicamente i pesi interi associati agli archi del grafo.\\
È possibile scegliere solo pesi interi positivi, ovvero i valori minimo e massimo del range devono essere compresi tra \(0 \geq minimo \geq massimo \geq n\), con \(n\to\infty\).\\

L'utente inoltre dovrà inserire il nome del grafo da creare il quale sarà utilizzato dal programma per creare la sotto porzione di filesystem relativa al grafo generato, ovvero nome-scelto.dir/nomescelto.edgelist, in modo da non creare conflitti tra i dati.\\

A questo punto il programma procederà a scrivere sul filesystem la struttura cartella/file.edgelist relativa al grafo appena creato.\\

Il file nel quale è codificato il grafo generato ha estensione .edgelist (il file .dot come specificato sopra è stato utilizzato solo in fase di debug e quindi non viene considerato nella trattazione).\\

Il file .edgelist è un file di tipo testuale che codifica sotto forma di lista (nodo, nodo, peso) la matrice di adiacenza che descrive il grafo generato.\\
Il file si presenta nella forma qui di seguito (ciascuna coppia (nodo, nodo) codifica un arco al quale viene associato il terzo valore della tupla, ovvero il peso intero).\\

\begin{description}
	\item nodo nodo peso
	\item 0    1    9
	\item 0    2    20
	\item 1    2    15
	\item 2    3    3
	\item ...
\end{description}

I nodi sono generati in modo totalmente dinamico utilizzando valori da 0 a n interi positivi, con n scelto dall'utente in fase di creazione quando richiesto come parametro associato alla tipologia di grafo da generare.\\

Per completezza specifichiamo che la libreria offre i seguenti 4 tipi di grafi.

\begin{itemize}
	\item Graph (grafo senza parallelismo non-orientato)
	\item Directed Graph (grafo senza parallelismo orientato)
	\item MultiGraph (grafo con parallelismo non-orientato)
	\item Directed MultiGraph (grafo con parallelismo orientato)
\end{itemize}

Nel programma è stata implementata solo la tipologia Graph, cioè grafi senza parallelismo non-orientati (con pesi interi positivi associati agli archi e valori interi positivi associati ai nodi) come richiesto dal problema modellato in precedenza, ovvero il gioco della k-colorazione generalizzata.\\

Il programma termina a questo punto l'esecuzione permettendo all'utente di scegliere se disegnare il grafo appena creato attraverso la libreria MatPlotLib, l'utente può saltare l'esecuzione di questa operazione rispondendo negativamente alla richiesta.\\

\subsection{generator.py : MULTIPLE MODE}
\justify
Il funzionamento della modalità MULTIPLE MODE, accessibile selezionando l'opzione MULTIPLE durante la scelta della classe, è quasi del tutto analogo a quello descritto in precedenza, le uniche differenze sono le seguenti.\\

Dopo aver scelto la tipologia di grafo da generare, all'utente verrà chiesto come parametro aggiuntivo il numero di iterazioni relativo all'algoritmo di generazione multipla.\\
In sostanza l'utente dovrà scegliere quanti grafi generare (appartenenti alla tipologia scelta in precedenza).\\

In seguito l'utente dovrà inserire un ulteriore parametro aggiuntivo, ovvero il valore minimo e massimo relativi al range all'interno del quale dovrà oscillare randomicamente il numero di nodi dei grafi da generare.\\
I valori minimo e massimo dovranno ovviamente essere compresi tra \(0 \geq minimo \geq massimo \geq n\), con \(n\to\infty\), come specificato in precedenza.\\

All'utente verrà inoltre chiesto di inserire il nome della cartella nella quale inserire gli output delle generazioni multiple, come nel caso descritto nella sezione SINGLE MODE.\\

A questo punto l'algoritmo di generazione provvederà a creare n grafi (con n scelto dall'utente, come descritto poco sopra) appartenenti alla tipologia scelta (ad esempio complete graph).\\
Ciascun grafo creato avrà un valore randomico di nodi compreso tra \(0 \geq minimo \geq numero di nodi \geq massimo \geq n\).\\

Il programma provverà automaticamente e in modo totalmente dinamico a creare le strutture nel filesystem necessarie a salvare correttamente i file .edgelist relativi a ciascun grafo creato durante la generazione multipla.\\
È stato implementato un processo di assegnamento automatico del nome per ciascun output generato che utilizza il round di iterazione assieme alla tipologia e al numero di nodi del grafo creato, in modo da evitare conflitti, inconsistenza e perdita dei dati.\\

Passiamo ora a descrivere il programma reader.py, il lettore di grafi, nel quale sono includi gli algoritmi per il calcolo degli equilibri di Nash e per il calcolo degli ottimi utilizzando le 2 funzioni di benessere sociale utilitario e egalitario, il vero cuore dell'implementazione.\\


\section{reader.py : il lettore di grafi}
\justify
il programma reader.py è un lettore dinamico di grafi anche esso scritto interamente in linguaggio Python che consente la lettura e la manipolazione dei grafi creati in modo asincrono attraverso l'utilizzo del generatore generator.py.\\

Inoltre il programma consente la modellazione del problema del gioco della k-colorazione generalizzata attraverso un massiccio uso della Standard Library del linguaggio Python.\\

Nel reader.py sono incluse le definizioni degli algoritmi per il calcolo degli equilibri di Nash e quelli per il calcolo dell'ottimo con funzione di benessere utilitario e egalitario.\\
Il programma permette la lettura di un singolo o di molteplici grafi per volta e consente l'esecuzione dei suddetti algoritmi sulle istanze lette.\\

Per rendere meno faticoso il lavoro relativo alla sperimentazione il programma è in grado di raccogliere efficacemente e organizzare i dati derivanti dalle esecuzioni in tempo reale.\\

In modo totalmente dinamico il lettore è in grado di salvare i risultati delle sperimentazioni manipolando le strutture relative al filesystem.\\
Come per il generatore il programma utilizza principalmente e in modo quasi esclusivo la libreria standard del linguaggio Python per eseguire queste operazioni.\\
Anche in questo caso le primitive e le funzioni utilizzate garantiscono la il funzionamento cross-platform e dunque la portabilità del programma.\\ Il tutto è stato ampiamente testato su macchine con sistema POSIX e non, il programma è totalmente ottimizzato e funzionale in questo senso.\\

I file di output per ciascuna esecuzione sono 2, il file .init (file testuale che contiene il salvataggio delle informazioni basilari del grafo, creato dopo aver effettuato la modellazione del problema) e il file.out (file testuale che contiene i risultati finali derivanti dall'esecuzione di uno o più algoritmi).\\

Sia per esecuzioni singole che multiple il risultato prodotto sarà sempre e comunque una coppia file.init / file.out.\\  

Anche in questo caso sono disponibili per l'utente moduli per il disegno dei grafi, in particolare pre e post modellazione e per rappresentare i risultati delle esecuzioni (ad esempio le colorazioni stabili ottenute dal calcolo degli equillibri di Nash).\\
La principale libreria in questo senso è MatPlotLib (la libreria PyGraphViz è stata utilizzata in fase di debug come specificato in precedenza).\\

Per non tediare il lettore confermiamo che anche il programma reader.py fa uso delle medesime librerie addizionali utilizzate nel programma generator.py, il loro uso è però marginale e dunque non verranno trattate, come in precedenza.\\
L'utilizzo più intesivo è relativo alle librerie Pick, per fornire un interfaccia di selezione da console all'utente, e alla libreria Colorama, utilizzata in fase di lettura e debug per colore l'output su console e mettere in risalto i dati fondamentali relativi alle esecuzioni degli algoritmi e alle operazioni essenziali di gestione del programma.\\

Di fondamentale importanza è l'utilizzo della libreria NetworkX che consente la lettura e l'interpretazione delle informazioni relative ai file testuali di creazione .edgelist e la ricostruzione in modo asincrono dei grafi generati in precedenza, in modo da utilizzarli per la sperimentazione.\\

Anche in questo caso qui di seguito vengono riportate le caratteristiche del programma assieme ad un esempio generale di un caso d'uso in modo da descrivere il funzionamento del programma e le operazioni disponibili per l'utente

Sono disponibili per l'utente, in modo speculare e simmetrico rispetto alla definizione del programma generator.py, 2 modalità di funzionamento per la lettura e l'esecuzione, la modalità SINGLE EXEC e la modalità MULTIPLE EXEC.\\
All'avvio del programma l'utente dovrà scegliere una delle 2 modaliltà d'uso attraverso un'interfaccia si selezione da console implementata attraverso l'utilizzo della libreria Pick.\\


\subsection{reader.py : SINGLE EXEC}
\justify
In modalità SINGLE EXEC il programma effettuerà una ricerca globale automatica e dinamica dei file con estensione .edgelist all'interno della cartella gen, la cartella relativa alle generazioni di grafi singoli creata dal generatore generator.py in modalità SINGLE MODE.\\

A questo punto l'utente dovrà selezione il grafo da modellare e sul quale eseguire una sperimentazione utilizzando gli algoritmi forniti all'interno del programma.\\
La scelta è implementata attraverso un'interfaccia di selezione creata con la libreria Pick nella quale apparirà la lista di file .edgelist trovati.\\

Il programma leggerà e interpreterà il contenuto del file.edgelist relativo al grafo selezionato dall'utente e attraverso le funzioni della libreria NetworkX ricostruire l'istanza dell'oggetto matematico desiderato creato in precedenza e sul quale si vuole iniziare una sperimentazione.\\

Ovviamente l'operazione genererà sempre e comunque un'istanza di grafo senza parallelismo non-orientato e con valori interi positivi associati ai nodi e pesi interi positivi associati agli archi (parametri scelti in precedenza, durante la fase di creazione del grafo e ricostruiti dalle funzioni di libreria).\\

In seguito l'utente sarà nuovamente chiamato a scegliere 2 parametri fondamentali per la modellazione del problema, ovvero il numero di colori e valore massimo per il profitto associato a questi ultimi.\\

L'utente potrà scegliere il numero massimo di colori presenti all'interno dell'istanza che si sta modellando.\\
Il numero massimo di colori non può essere superiore al numero di nodi del grafo, altrimenti la sperimentazione perderebbe di senso e in più non verrebbe rispettata la definizione del problema presentata nel Capitolo 2.\\
I colori, come ogni altro parametro del quale non è richiesto un range (valore minimo e massimo), saranno numerati da 0 a n, con n scelto dall'utente e sempre \(n \leq numero di nodi del grafo\).\\

Come struttura dati è stata scelta la lista poichè l'elenco dei colori non necessità di essere manipolato in maniera complessa, non vi è inoltre l'esigenza di effettuare query di ricerca indicizzate e altre operazioni poco efficaci sulle liste.\\
Semplicemente la lista dei colori andrà iterata più volte e in modo innestato durante i vari cicli degli algoritmi implementati.\\
Per la creazione è stata utilizzata la potente list comprehension interna a al linguaggio Python.\\

L'utente inoltre potrà scegliere il valore massimo all'interno del quale oscilleranno randomicamente i profitto associati ai colori per ciascun giocatore.\\
In questo caso non vi sono restrizioni, scelto n, i valori oscilleranno in modo randomico tra 0 e n.\\
I profitti legati ai colori, per definizione del modello potranno essere differenti da giocatore a giocatore, ad esempio il giocatore 0 avrà un profitto di 56 per il colore 0, il giocatore 1 avrà un profitto di 29 per il colore 0 (i valori sono totalmente casuali e servono solo da esempio).\\

Proprio per questo motivo la struttura dati implementata per soddisfare le caratteristiche del modello è un doppio dizionario innestato.\\
Assieme alla struttura relativa alle coppie arco-peso e quella relativa all'associazione nodo-colore che presenteremo in seguito, quest'ultima rappresenta uno dei componenti più utilizzati del programma.\\
È necessaria in questo caso un'implementazione che permetta l'utilizzo di query di ricerca asincrone a doppia chiave (l'operazione più pesante che interessa questa struttura).\\
È stata utilizzata a questo scopo la potente nested dictionary comprehension del linguaggio Python, in modo da eliminare possibilità di conflitto e inconsistenza dei dati, in modo da disambiguare le entry del dizionario in modo efficace e dunque in modo da garantire l'utilizzo di interrogazioni ad accesso diretto utili nel programma.\\
È stata generata la seguente struttura (esempio).

\begin{description}
	\item nodo 0 :: colore 0 : profitto 24, colore 1 : profitto 38, colore 2 : profitto 30, ...
	\item nodo 1 :: colore 0 : profitto 11, colore 1 : profitto 55, colore 2 : profitto 23, ...
	\item nodo 2 :: colore 0 : profitto 14, colore 1 : profitto 65, colore 2 : profitto 26, ...
	\item nodo 3 :: colore 0 : profitto 99, colore 1 : profitto 66, colore 2 : profitto 86, ...
	\item ...
\end{description}

A questo punto viene generata automaticamente la colorazione iniziale per il grafo corrente.\\
Il colore è stato gestito attraverso la struttura dati del dizionario, ad ogni nodo viene associata una singola entry chiave : valore come etichetta, la chiave rappresenta il nome del parametro, ovvero la stringa "color" e il valore rappresenta il colore con il quale è colorato in nodo.\\
Le etichette sono gestite a livello di libreria NetworkX, dunque è stato seguito lo standard esplicato in documentazione.\\
È dunque possibile accedere e manipolare, con facilità e alta velocità computazionale, i dati all'interno dei dizionari che rappresentano i parametri associati al nodo attravero semplici query ad accesso diretto e a singola chiave, ovvero il parametro "color" nel nostro caso.\\
La colorazione iniziale è del tipo seguente.

\begin{description}
	\item nodo 0 : colore 2
	\item nodo 1 : colore 1
	\item nodo 2 : colore 2
	\item nodo 3 : colore 0
	\item ...
\end{description}

Tutti i dati relativi alla modellazione iniziale dell'istanza letta per ricreare le condizioni di partenza del nostro gioco sono salvate nel file.init correlato all'esecuzione corrente.\\
Specifichiamo nuovamente l'assoluta gestione cross-platform del filesystem e l'efficente ottimizzazione eseguita sul lato input / output su file, il tutto è affiancato ovviamente da numerosissimi test su diverse tipologie di macchine.\\

Nel file testuale .init relativo all'esecuzioni vengono inseriti i seguenti parametri.

\begin{itemize}
	\item lista dei colori
	\item colorazione iniziale del grafo
	\item numero di nodi del grafo
	\item profitti associati ai colori per ogni giocatore
	\item lista degli archi del grafo con pesi associati
	\item numero degli archi del grafo
\end{itemize}

È importante specificare che il peso degli archi è gestito in modo del tutto identico a quello relativo al colore dei nodi.\\
Infatti il peso (e gli eventuali altri parametri implentabili in modo nativo grazie alla libreria NetworkX) degli archi si presenta come un dizionario chiave : valore, nel quale la chiave corrisponde al nome del parametro, ovvero la stringa "weight" e il valore rappresenta il vero e proprio peso associato all'arco.\\

Anche qui dunque una struttura dati di questo tipo permette un'alta velocità di calcolo e recupero delle informazioni con query ad accesso diretto a chiave singola, ovvero il nome del parametro richiesto associato all'arco che nel nostro caso è la stringa "weight".\\
La struttura dati si presenta come segue.

\begin{description}
	\item nodo 0  nodo 1  weight : 34
	\item nodo 1  nodo 2  weight : 55
	\item nodo 2  nodo 5  weight : 5
	\item nodo 3  nodo 2  weight : 23
	\item ...
\end{description}

Terminata la creazione del file testuale con estensione .init il programma procede e genera tutte le strutture del filesystem necessarie per salvare correttamente i risultati della sperimentazione corrente.\\
In particolare crea una nuova cartella all'interno della direcotry result (poichè siamo in SINGLE EXEC) in modo dinamico prendendo il nome del grafo corrente importato dall'utente.\\
Assegna il medesimo nome al file .init (e al file .out che vedremo tra poco) che viene generato all'interno del path appena generato.\\

A questo punto l'utente viene chiamato a compiere un'ennesima scelta, l'ultima relativa al flusso di esecuzione corrente.\\
All'utente viene chiesto di scegliere tra le 3 opzioni di calcolo disponibili, ciascuna corrispondete a 1 dei 3 algoritmi implementati.

\begin{itemize}
	\item cacolo della colorazione ottima (ottimo con funzione di benessere sociale utilitaria)
	\item cacolo della colorazione ottima (ottimo con funzione di benessere sociale egalitaria)
	\item cacolo della colorazione stabile (equilibrio di Nash)
	\item esci
\end{itemize}

Vi inoltre un'ulteriore opzione per uscire dal programma, prima dell'esecuzione che potrebbe essere pesante a livello di calcolo e dunque lenta al livello temporale, soprattutto per ciò che concerne il calcolo degli ottimi.\\

Per l'utente che esegue in modalità SINGLE EXEC, questo punto del programma rappresenta un hub di esecuzione.\\
L'utente infatti può eseguire, quante volte vuole, i vari algoritmi (può anche eseguirli tutti sull'istanza corrente o uscire e non fare nulla come specificato poco sopra).\\
Terminata l'esecuzione di uno degli algoritmi, all'utente verrà lasciato il pieno controllo.\\
Quest'ultimo potrà leggere il risultato dell'esecuzione da console e analizzare i vari cicli eseguiti e poi tornare all'hub nel quale potrà avviare una nuova esecuzione o uscire dal programma.\\

Prima di affrontare la descrizione degli algoritmi, il vero cuore dell'implementazione, analizziamo al modalità di funzionamento MULTIPLE EXEC relativa al programma reader.py.\\

\subsection{reader.py : MULTIPLE EXEC}
\justify
Per non tediare il lettore, possiamo specificare che la modalità MULTIPLE EXEC esegue le medesime operazioni di preparazione e modellazione delle varie istanze e salva il risultato di tale operazione in un file testuale estensione .init.\\

Una prima differenza sostanziale è relativa al path in cui vengono cercati i file .edgelist nella fase di scelta della modalità di esecuzione del lettore.\\
Scegliendo MULTIPLE EXEC il programma effettuerà una ricerca globale di tutte le cartelle presenti all'interno della cartella mgen, ovvero la cartella contenente tutti i risultati delle varie generazioni multiple, creata dinamicamente dal generatore di grafi in modalità MULTIPLE MODE.\\

A questo punto l'utente sarà nuovamente interpellato e attraverso un'interfaccia di selezione da console dovrà selezionare la cartella contenente i grafi sui quali desidera fare sperimentazione.\\
Una volta selezionata la cartella il programma utilizzerà tutti i file con estensione .edgelist come istanze dell'esecuzione multipla.\\
In particolare il programma si sviluppa all'interno di un ciclo principale che scorrè tutti i file con estensione .edgelist presenti nella cartella scelta dall'utente.\\

Il ciclo prende un grafo alla volta, opera la fase di pre-esecuzione modellando l'istanza corrente secondo la definzione del problema e poi esegue l'algoritmo per il calcolo della colorazione stabile trovando l'equilibrio di Nash per il grafo corrente.\\
Il ciclo e l'esecuzione, che potrebbe coinvolgere un numero molto grande di grafi, avvengono in maniera totalmente automatica e dunque l'utente può disinteressarsi del programma poichè slegato dal flusso di esecuzione.\\




%\section{Analisi e descrizione degli algoritmi}
%\justify
%
%\subsection{nash_equilibrium : l'algoritmo per il calcolo degli equilibri di Nash}
%\justify
%
%\subsection{utilitarian_social_welfare : l'algoritmo per il calcolo dell'ottimo relativo alla funzione di benessere sociale utilitario}
%\justify
%
%\subsection{egalitarian_social_welfare : l'algoritmo per il calcolo dell'ottimo relativo alla funzione di benessere sociale egalitario}
%\justify

	\part{Sperimentazione}
	\chapter{Sperimentazione}
\justify
In questo capitolo esplicheremo e descriveremo gli aspetti e le caratteristiche generali relative all'attività di sperimentazione.\\

Inizieremo descrivendo in breve i moduli relativi alla generazione e alla lettura di grafi randomici situati all'interno dei 2 programmi implementati : generator.py e reader.py.\\

Procederemo delineando un elenco delle assunzioni e delle decisioni compiute che riguardano in generale la modellazione del problema e nello specifico la costruzione e la tipologia dei grafi implementati.\\
Queste rappresentano la base di partenza sulla quale è stata portata avanti l'attività di sperimentazione e sono fondamentali per comprendere meglio la parte concettuale e gli obiettivi dietro quest'ultima.\\

L'ultima fase prevede la presentazione delle conclusioni associate ai risultati ottenuti, in relazione al problema in oggetto (gioco della k-colorazione generalizzata) e in relazione alla trattazione matematica e ai teoremi delineati all'interno del documento alla base di questo studio (Generalized Graph k-Coloring Games).\\

\section{Obiettivi}
\justify
Si specifica, in breve, che l'obiettivo primario dell'intera sperimentazione è quello di calcolare, misurare e analizzare il numero di passi compiuti dall'algoritmo ottimizzato per il calcolo degli equilibri di Nash sulle varie istanze di gioco della $k$-colorazione generalizzata modellate attraverso l'utilizzo di grafi randomici.\\
A questo scopo ricordiamo che il problema di trovare un equilibrio di Nash per il gioco della $k$-colorazione generalizzata è PLS-Completo e dunque è interessante analizzare il numero di miglioramenti effettuati relativi alle varie dinamiche trovate sulle varie istanze studiate.
Inoltre specifichiamo che è stato svolto uno studio parallelo riguardante la validità della colorazione stabile in termini di benessere sociale utilitario ed egalitario. A questo proposito sono stati utilizzati i 2 algoritmi per il calcolo dell'ottimo con funzione di benessere sociale utilitario e egalitario. I suddetti algoritmi non prevedono alcuna strategia di ottimizzazione legata alle iterazioni e utilizzano un approccio a forza bruta che rende la computazione molto complessa. Ricordiamo che il problema di calcolare gli ottimi con funzioni di benessere sociale utilitario e egalitario appartiene alla classe NP.\\
Per descrivere la lontananza del valore di benessere sociale utilitario o egalitario della colorazione stabile, dal valore di benessere sociale utilitario o egalitario della colorazione ottima, sono state utilizzate le definizioni di prezzo dell'anarchia sperimentale utilitario e egalitario, presentate in precedenza.\\

\section{gnp\_random\_graph (grafo di Erdős-Rényi o grafo binomiale)}
\justify
Il principali oggetti matematici studiati durante l'attività di sperimentazione sono i grafi randomici.\\

In particolare è stata dedicata una porzione del codice all'interno del programma generator.py per la creazione e la manipolazione di una specifica tipologia di grafo : il gnp\_random\_graph.\\

Il gnp\_random\_graph, conosciuto anche come grafo di Erdős-Rényi o grafo binomiale, è l'oggetto attorno al quale ruota l'intera attività di implementazione.\\ Quest'ultimo appartiene alla classe Random Graphs.\\
Tale classe è contenuta all'interno della libreria di generazione e manipolazione di grafi NetworkX, utilizzata durante l'attività di implementazione, come specificato nella sezione precedente.\\
La tipologia di grafo gnp\_random\_graph, come delineato in precedenza, possiede dunque un costruttore di classe all'interno della suddetta libreria.\\

Il costruttore di classe, opportunamente parametrizzato in modo automatico oppure grazie all'intervento attivo dell'utente, è stato utilizzato per la creazione delle varie istanze di questa tipologia di grafo.\\

L'esecuzione dell'algoritmo di generazione comporta un complessità temporale pari a \(O(n^2)\), dunque un valore accettabile.\\
Si cita in questo senso la presenza di algoritmi di generazione più rapidi correlati a differenti costruttori, come ad esempio l'algoritmo relativo alla tipologia fast\_gnp\_random\_graph, che, per valori piccoli di $p$ (probabilità di creare archi tra coppie di nodi), riesce a generare grafi sparsi in \(O(n+m)\).\\

Dato che la medio-alta densità dei grafi rende più interessante lo studio trattato, è stato scelto di tralasciare questo tipo di costruttore alternativo e di utilizzare solo ed esclusivamente il costruttore gnp\_random\_graph per la generazione.\\

Descriviamo ora le modalità di generazione per le esecuzioni in modalità SINGLE EXEC e in modalità MULTIPLE EXEC e procediamo con la delineazione delle assunzioni formulate durante la modellazione dei grafi creati.\\

\subsection{Generazione gnp\_random\_graph)}
\justify
Le funzioni per la generazione di grafi gnp\_random\_graph sono raggiungibili all'interno del programma generator.py selezionando la modalità di esecuzione MULTIPLE MODE e selezionando in seguito la classe gnp\_random\_graph.\\

All'utente è richiesto, come primo parametro, l'inserimento del numero di iterazioni da compiere, che corrispondono al numero di grafi da costruire all'interno del singolo processo di generazione corrente.\\

In seguito l'utente dovrà inserire il numero di nodi per i grafi da generare.\\
Tale valore, per assunzione, sarà un valore fisso per ciascun ciclo di generazione e dunque applicato a tutti i grafi appartenenti a quest'ultimo.\\

Gli ultimi parametri da inserire sono i valori massimo e minimo all'interno dei quali oscilleranno randomicamente i pesi associati agli archi del grafo, il tutto avviene nelle modalità specificate nella sezione relativa al generatore.\\

A questo punto l'utente sarà chiamato a scegliere la sotto modalità di generazione dei grafi, la scelta della modalità "single" comporterà la creazione di un path specifico all'interno della cartella gen nel quale verranno salvati i vari risultati della creazione seguendo lo schema presentato durante la descrizione della modalità SINGLE MODE relativa al generatore di grafi.\\
La scelta della modalità "multiple" comporterà la creazione di un path specifico all'interno della cartella mgen nel quale verranno salvati i vari risultati della creazione seguendo lo schema presentato durante la descrizione della modalità MULTIPLE MODE relativa al generatore di grafi. \\

L'ultimo valore, gestito in modo randomico per ciascuna istanza della generazione corrente, è $p$.\\
Il parametro $p$ rappresenta la probabilità di generare un archi tra le varie coppie di nodi del grafo corrente, viene fatto oscillare in modo randomico tra i valori \(0 \geq p \geq 1\).\\
La variabile $p$ conterrà dunque un valore flottante (tipo float) la cui precisione è gestita in modo automatico e impostata a 2, ovvero sono ammesse fino a 2 cifre dopo la virgola.\\

Tralasciando i dettagli trascurabili descritti all'interno della sezione relativa all'implementazione, specifichiamo che la manipolazione del filesystem finalizzata al salvataggio degli output di generazione è effettuata in modo automatico e dinamico dal programma.\\
Quest'ultima è ottimizzata, cross-platform e presenta, come descritto in precedenza, primitive e funzioni appartenenti alla sola Standard Library del linguaggio Python.\\
Ovviamente presenta le medesime caratteristiche dei moduli utilizzati e descritti in precedenza e dunque vengono accuratamente gestiti e evitati possibili conflitti tra i dati e problemi di inconsistenza.\\

\subsection{Assunzioni generali}
\justify
In breve, durante l'attività di progettazione e programmazione riguardanti la sperimentazione, sono state effettuate alcune decisioni relative a quest'ultima e sono state formulate le seguenti assunzioni sulla base delle quali si è svolta l'intera analisi.\\

\begin{itemize}
	\item I risultati delle varie sessioni di sperimentazione sono presentati al lettore sotto forma di tabelle riassuntive correlate da semplici leggende riguardanti la definizione dei parametri al fine di rendere più chiara e semplice la lettura e la visione dei dati raccolti
	\item Ogni sperimentazione si è svolta fissando il numero di nodi e facendo oscillare randomicamente gli altri parametri come ad esempio il numero di colori $k$ oppure la probabilità relativa alla densità del grafo $p$.\\
	\item Il parametro $p$ relativo alla densità di un grafo, è un valore flottante a precisione 2 (2 cifre dopo la virgola) che oscilla tra i valori \(0 \geq p \geq 1\) e che descrive la probabilità di creare un arco tra un coppia di nodi e dunque definisce quanto il grafo in oggetto sia sparso (o denso)
	\item Il parametro $k$ definisce il numero di colori utilizzati per generare le differenti colorazioni, ciascuna delle quali rappresenta uno stato del gioco in oggetto. Ciascun colore \(i \in K\), con \(K = 1,\ldots,k\) (il set di colori), rappresenta una strategia per un giocatore. Per convenzione (la sperimentazione effettuata perderebbe di senso altrimenti) il parametro $k$ deve essere \(k \leq n\), con $n$ uguale al numero di nodi del grafo corrente
	\item Per ogni sessione di sperimentazione sono elencati 10 risultati rilevanti 
	\item Per le esecuzioni più pesati è stato concesso un limite temporale più ampio, ad esempio da 1 minuto a 2 minuti
	\item Il range all'interno del quale oscillano dinamicamente i pesi degli archi, per ciascuna esecuzione, è definito per convenzione \(0 \geq a \geq 100\), con $a$ uguale ai pesi degli archi
	\item Il range all'interno del quale oscillano i valore dei profitti associati ai colori per ciascun giocatore, per ciascuna esecuzione, è definito per convenzione \(0 \geq p \geq 100\), con $p$ uguale ai profitti associati ai colori per ogni giocatore
\end{itemize}

\subsection{Tipologie di sperimentazione}
\justify
Le tipologie di sperimentazione effettuate sono le seguenti

\begin{itemize}
	\item [\textbf{Tipologia I}] vengono eseguite in sequenza le seguenti operazioni su istanze di dimensione modesta, a causa dell'enorme dispendio temporale di alcune operazioni :
	\begin{enumerate}
		\item calcolo dell'ottimo relativo alla funzione di benessere sociale utilitario con limitatore temporale
		\item calcolo dell'ottimo relativo alla funzione di benessere sociale egalitario con limitatore temporale
		\item calcolo della colorazione stabile relativa alla definizione di equilibrio di Nash senza limitatore temporale
		\item definizione del valore di benessere sociale utilitario relativo alla colorazione stabile
		\item definizione del valore di benessere sociale egalitario relativo alla colorazione stabile
		\item definizione del valore relativo al prezzo dell'anarchia sperimentale utilitario
		\item definizione del valore relativo al prezzo dell'anarchia sperimentale egalitario
	\end{enumerate}
	\item [\textbf{Tipologia II}] vengono eseguite in sequenza le seguenti operazioni su istanze di dimensione medio-grande :
	\begin{enumerate}
		\item calcolo della colorazione stabile relativa alla definizione di equilibrio di Nash con limitatore temporale
		\item definizione del valore di benessere sociale utilitario relativo alla colorazione stabile
		\item definizione del valore di benessere sociale egalitario relativo alla colorazione stabile
	\end{enumerate}
\end{itemize}

\section{Risultati sperimentazione - Tipologia I}
\justify
Qui di seguito vengono presentati i risultati relativi alla sperimentazione di Tipologia I.\\

\textbf{LEGGENDA : }

\begin{itemize}
	\item Il parametro \textbf{k} descrive il numero di colori relativo all'istanza corrente
	\item Il parametro \textbf{p} descrive la probabilità relativa alla densità dell'istanza corrente
	\item Il parametro \textbf{a} descrive il numero di archi dell'istanza corrente
	\item Il valore \textbf{o\_usw} è il risultato ottenuto dall'applicazione, sull'istanza corrente, del calcolo dell'ottimo con funzione di benessere sociale utilitario
	\item Il valore \textbf{o\_esw} è il risultato ottenuto dall'applicazione, sull'istanza corrente, del calcolo dell'ottimo con funzione di benessere sociale egalitario
	\item Il valore \textbf{n\_step} è il numero di step (mosse migliorative effettuate) ottenuto dall'applicazione, sull'istanza corrente, del calcolo della colorazione stabile seguendo la definizione di equilibrio di Nash
	\item Il valore \textbf{n\_usw} è il risultato del calcolo del benessere sociale utilitario relativo alla colorazione stabile trovata per l'istanza corrente
	\item Il valore \textbf{n\_esw} è il risultato del calcolo del benessere sociale egalitario relativo alla colorazione stabile trovata per l'istanza corrente
	\item Il valore \textbf{u\_poa} è il risultato del calcolo del prezzo dell'anarchia utilitario sperimentale relativo all'istanza corrente
	\item Il valore \textbf{e\_poa} è il risultato del calcolo del prezzo dell'anarchia egalitario sperimentale relativo all'istanza corrente
	\item Il parametro \textbf{t} è valore, assegnato dinamicamente, del limitatore temporale (1 = 1 minuto, 2 = 2 minuti, ecc...), il valore vale per ciascun calcolo (assegnando \(t = 5\), la singola esecuzione completa dovrebbe durare circa 10-15 minuti in totale)
\end{itemize}


\subsection{Tipologia I - Grafi Random - 3 nodi [$n=3$]}

\begin{table}[H]
\centering
\scalebox{0.9} {
\begin{tabular}{|c|c|c|c|c|c|c|c|c|c|c|}
\hline
\textbf{k} & \textbf{p} & \textbf{a} & \textbf{o\_usw} & \textbf{o\_esw} & \textbf{n\_step} & \textbf{n\_usw} & \textbf{n\_esw} & \textbf{u\_poa} & \textbf{e\_poa} & \textbf{t} \\ \hline
3 & 0.90 & 2 & 496 & 126 & 4 & 496 & 126 & 1 & 1 & 1 \\ \hline
2 & 0.55 & 3 & 406 & 120 & 1 & 406 & 91 & 1 & 1.3118 & 1 \\ \hline
3 & 0.71 & 2 & 486 & 91 & 2 & 486 & 91 & 1 & 1 & 1 \\ \hline
2 & 0.68 & 3 & 425 & 87 & 2 & 425 & 81 & 1 & 1.0740 & 1 \\ \hline
2 & 0.46 & 2 & 468 & 97 & 2 & 468 & 97 & 1 & 1 & 1 \\ \hline
3 & 0.82 & 3 & 550 & 157 & 2 & 527 & 157 & 1.0436 & 1 & 1 \\ \hline
3 & 0.43 & 2 & 455 & 114 & 2 & 455 & 114 & 1 & 1 & 1 \\ \hline
3 & 0.36 & 3 & 625 & 178 & 2 & 549 & 170 & 1.1384 & 1.0470 & 1 \\ \hline
2 & 0.96 & 2 & 414 & 113 & 2 & 333 & 99 & 1.2432 & 1.1414 & 1 \\ \hline
2 & 0.14 & 3 & 548 & 146 & 3 & 548 & 146 & 1 & 1 & 1 \\ \hline
\end{tabular}
}
\caption{Tipologia I - Grafi Random - 3 nodi}
\label{tab:sperimentazione-tipo1-3nodi}
\end{table}


\subsection{Tipologia I - Grafi Random - 5 nodi [$n=5$]}

\begin{table}[H]
\centering
\scalebox{0.9} {
\begin{tabular}{|c|c|c|c|c|c|c|c|c|c|c|}
\hline
\textbf{k} & \textbf{p} & \textbf{a} & \textbf{o\_usw} & \textbf{o\_esw} & \textbf{n\_step} & \textbf{n\_usw} & \textbf{n\_esw} & \textbf{u\_poa} & \textbf{e\_poa} & \textbf{t} \\ \hline
5 & 0.90 & 9 & 314 & 45 & 5 & 314 & 41 & 1 & 1.0975 & 1 \\ \hline
4 & 0.27 & 5 & 485 & 94 & 4 & 485 & 94 & 1 & 1 & 1 \\ \hline
3 & 0.82 & 9 & 410 & 45 & 4 & 410 & 40 & 1 & 1.1250 & 1 \\ \hline
4 & 0.14 & 4 & 394 & 40 & 4 & 394 & 40 & 1 & 1 & 1 \\ \hline
2 & 0.79 & 7 & 420 & 37 & 5 & 420 & 37 & 1 & 1 & 1 \\ \hline
5 & 0.65 & 9 & 497 & 86 & 6 & 497 & 86 & 1 & 1 & 2 \\ \hline
3 & 0.40 & 6 & 805 & 120 & 5 & 766 & 120 & 1.0509 & 1 & 1 \\ \hline
3 & 0.52 & 5 & 932 & 85 & 4 & 932 & 85 & 1 & 1 & 1 \\ \hline
5 & 1 & 10 & 913 & 149 & 3 & 913 & 149 & 1 & 1 & 2 \\ \hline
2 & 0.24 & 3 & 590 & 64 & 3 & 566 & 64 & 1.0402 & 1 & 1 \\ \hline
\end{tabular}
}
\caption{Tipologia I - Grafi Random - 5 nodi}
\label{tab:sperimentazione-tipo1-5nodi}
\end{table}

\subsection{Tipologia I - Grafi Random - 5 nodi [$n=5$] e 3 colori [$k=3$]}

\begin{table}[H]
\centering
\scalebox{0.9} {
\begin{tabular}{|c|c|c|c|c|c|c|c|c|c|c|}
\hline
\textbf{p} & \textbf{a} & \textbf{o\_usw} & \textbf{o\_esw} & \textbf{n\_step} & \textbf{n\_usw} & \textbf{n\_esw} & \textbf{u\_poa} & \textbf{e\_poa} & \textbf{t} \\ \hline
0.40 & 5 & 1064 & 148 & 6 & 1064 & 126 & 1 & 1.1746 & 1 \\ \hline
0.50 & 7 & 1113 & 100 & 3 & 1019 & 76 & 1.0922 & 1.3157 & 1 \\ \hline
0.60 & 6 & 1253 & 163 & 8 & 1121 & 163 & 1.1177 & 1 & 1 \\ \hline
0.70 & 5 & 862 & 95 & 7 & 862 & 95 & 1 & 1 & 1 \\ \hline
0.80 & 9 & 1104 & 166 & 9 & 1061 & 125 & 1.0405 & 1.328 & 1 \\ \hline
0.90 & 9 & 1249 & 210 & 7 & 1249 & 173 & 1 & 1.1475 & 1 \\ \hline
1 & 10 & 1564 & 261 & 3 & 1451 & 246 & 1.0778 & 1.0609 & 1 \\ \hline
\end{tabular}
}
\caption{Tipologia I - Grafi Random - 5 nodi e 3 colori}
\label{tab:sperimentazione-tipo1-5nodi3colori}
\end{table}

\subsection{Tipologia I - Grafi Random - 5 nodi [$n=5$] e probabilità $0.70$ [$p=0.70$]}

\begin{table}[H]
\centering
\scalebox{0.9} {
\begin{tabular}{|c|c|c|c|c|c|c|c|c|c|c|}
\hline
\textbf{k} & \textbf{a} & \textbf{o\_usw} & \textbf{o\_esw} & \textbf{n\_step} & \textbf{n\_usw} & \textbf{n\_esw} & \textbf{u\_poa} & \textbf{e\_poa} & \textbf{t} \\ \hline
2 & 6 & 794 & 118 & 3 & 778 & 77 & 1.0205 & 1.5324 & 1 \\ \hline
3 & 6 & 930 & 148 & 6 & 912 & 120 & 1.0197 & 1.2333 & 1 \\ \hline
4 & 6 & 1015 & 152 & 6 & 1015 & 152 & 1 & 1 & 1 \\ \hline
5 & 6 & 1063 & 149 & 4 & 1063 & 149 & 1 & 1 & 1 \\ \hline
\end{tabular}
}
\caption{Tipologia I - Grafi Random - 5 nodi e probabilità $0.70$}
\label{tab:sperimentazione-tipo1-5nodi070probab}
\end{table}

\subsection{Tipologia I - Grafi Random - 7 nodi [$n=7$]}

\begin{table}[H]
\centering
\scalebox{0.9} {
\begin{tabular}{|c|c|c|c|c|c|c|c|c|c|c|}
\hline
\textbf{k} & \textbf{p} & \textbf{a} & \textbf{o\_usw} & \textbf{o\_esw} & \textbf{n\_step} & \textbf{n\_usw} & \textbf{n\_esw} & \textbf{u\_poa} & \textbf{e\_poa} & \textbf{t} \\ \hline
3 & 0.36 & 9 & 1359 & 75 & 4 & 1359 & 60 & 1 & 1.25 & 2 \\ \hline
3 & 0.75 & 13 & 1922 & 206 & 5 & 1723 & 158 & 1.1154 & 1.3037 & 5 \\ \hline
3 & 0.54 & 11 & 1819 & 118 & 5 & 1694 & 118 & 1.0737 & 1 & 5 \\ \hline
2 & 0.89 & 18 & 1414 & 161 & 7 & 1361 & 117 & 1.0389 & 1.3760 & 2 \\ \hline
3 & 0.27 & 8 & 1456 & 151 & 7 & 1323 & 117 & 1.1005 & 1.2905 & 2 \\ \hline
4 & 0.39 & 13 & 1562 & 138 & 8 & 1539 & 138 & 1.0149 & 1 & 5 \\ \hline
4 & 0.40 & 9 & 1528 & 93 & 8 & 1496 & 93 & 1.0213 & 1 & 5 \\ \hline
5 & 0.54 & 14 & 2135 & 244 & 8 & 2135 & 244 & 1 & 1 & 10 \\ \hline
2 & 0.90 & 20 & 1876 & 218 & 7 & 1857 & 152 & 1.0102 & 1.4321 & 2 \\ \hline
3 & 0.61 & 13 & 1541 & 153 & 7 & 1525 & 134 & 1.0104 & 1.1417 & 5 \\ \hline
\end{tabular}
}
\caption{Tipologia I - Grafi Random - 7 nodi}
\label{tab:sperimentazione-tipo1-7nodi}
\end{table}


\subsection{Tipologia I - Grafi Random - 10 nodi [$n=10$]}

\begin{table}[H]
\centering
\scalebox{0.9} {
\begin{tabular}{|c|c|c|c|c|c|c|c|c|c|c|}
\hline
\textbf{k} & \textbf{p} & \textbf{a} & \textbf{o\_usw} & \textbf{o\_esw} & \textbf{n\_step} & \textbf{n\_usw} & \textbf{n\_esw} & \textbf{u\_poa} & \textbf{e\_poa} & \textbf{t} \\ \hline
2 & 0.77 & 41 & 3294 & 260 & 8 & 3155 & 241 & 1.0440 & 1.0728 & 5 \\ \hline
2 & 0.42 & 20 & 2281 & 127 & 7 & 2049 & 127 & 1.1132 & 1 & 5 \\ \hline
2 & 0.26 & 13 & 1786 & 77 & 6 & 1682 & 65 & 1.0618 & 1.1846 & 5 \\ \hline
2 & 0.99 & 45 & 3228 & 277 & 11 & 2935 & 247 & 1.0998 & 1.1214 & 5 \\ \hline
2 & 0.52 & 19 & 1842 & 108 & 7 & 1735 & 82 & 1.0616 & 1.3170 & 5 \\ \hline
3 & 0.47 & 20 & 2374 & 120 & 9 & 2348 & 120 & 1.0110 & 1 & 15 \\ \hline
3 & 0.23 & 11 & 2148 & 82 & 6 & 2075 & 82 & 1.0351 & 1 & 15 \\ \hline
3 & 0.24 & 10 & 1783 & 82 & 7 & 1783 & 64 & 1 & 1.2812 & 15 \\ \hline
3 & 0.32 & 18 & 2227 & 103 & 10 & 2203 & 93 & 1.0108 & 1.1075 & 15 \\ \hline
3 & 0.43 & 15 & 2310 & 131 & 8 & 2310 & 131 & 1 & 1 & 15 \\ \hline
\end{tabular}
}
\caption{Tipologia I - Grafi Random - 10 nodi}
\label{tab:sperimentazione-tipo1-10nodi}
\end{table}

\section{Risultati sperimentazione - Tipologia II}
\justify
Qui di seguito vengono presentati i risultati relativi alla sperimentazione di Tipologia II.\\

\textbf{LEGGENDA : }

\begin{itemize}
	\item Il parametro \textbf{k} descrive il numero di colori relativo all'istanza corrente
	\item Il parametro \textbf{p} descrive la probabilità relativa alla densità dell'istanza corrente
	\item Il parametro \textbf{a} descrive il numero di archi dell'istanza corrente
	\item Il valore \textbf{n\_step} è il numero di step (mosse migliorative effettuate) ottenuto dall'applicazione, sull'istanza corrente, del calcolo della colorazione stabile seguendo la definizione di equilibrio di Nash
	\item Il valore \textbf{n\_usw} è il risultato del calcolo del benessere sociale utilitario relativo alla colorazione stabile trovata per l'istanza corrente
	\item Il valore \textbf{n\_esw} è il risultato del calcolo del benessere sociale egalitario relativo alla colorazione stabile trovata per l'istanza corrente
	\item Il parametro \textbf{t} è valore, assegnato dinamicamente, del limitatore temporale (1 = 1 minuto, 2 = 2 minuti, ecc...)
\end{itemize}

\subsection{Tipologia II - Grafi Random - 15 nodi [$n=15$]}

\begin{table}[H]
\centering
\begin{tabular}{|c|c|c|c|c|c|c|}
\hline
\textbf{k} & \textbf{p} & \textbf{a} & \textbf{n\_step} & \textbf{n\_usw} & \textbf{n\_esw} & \textbf{t} \\ \hline
7 & 0.75 & 78 & 9 & 9398 & 365 & 1 \\ \hline
7 & 0.84 & 83 & 19 & 9664 & 457 & 1 \\ \hline
13 & 0.49 & 50 & 17 & 6837 & 176 & 1 \\ \hline
5 & 0.30 & 35 & 7 & 5226 & 133 & 1 \\ \hline
9 & 0.87 & 90 & 19 & 10747 & 586 & 1 \\ \hline
5 & 0.49 & 41 & 15 & 5206 & 174 & 1 \\ \hline
9 & 0.89 & 97 & 14 & 10937 & 451 & 1 \\ \hline
10 & 0.79 & 84 & 19 & 10437 & 582 & 1 \\ \hline
4 & 0.88 & 87 & 9 & 9295 & 473 & 1 \\ \hline
13 & 0.38 & 34 & 14 & 4185 & 161 & 1 \\ \hline
\end{tabular}
\caption{Tipologia II - Grafi Random - 15 nodi}
\label{tab:sperimentazione-tipo1-15nodi}
\end{table}

\subsection{Tipologia II - Grafi Random - 15 nodi [$n=15$] e 8 colori [$k=8$]}

\begin{table}[H]
\centering
\begin{tabular}{|c|c|c|c|c|c|c|}
\hline
\textbf{p} & \textbf{a} & \textbf{n\_step} & \textbf{n\_usw} & \textbf{n\_esw} & \textbf{t} \\ \hline
0.40 & 53 & 18 & 6199 & 213 & 1 \\ \hline
0.50 & 54 & 25 & 7243 & 338 & 1 \\ \hline
0.60 & 69 & 22 & 8747 & 459 & 1 \\ \hline
0.70 & 82 & 21 & 9376 & 414 & 1 \\ \hline
0.80 & 86 & 16 & 10292 & 556 & 1 \\ \hline
0.90 & 93 & 22 & 11063 & 569 & 1 \\ \hline
1 & 105 & 17 & 12055 & 604 & 1 \\ \hline
\end{tabular}
\caption{Tipologia II - Grafi Random - 15 nodi e 8 colori}
\label{tab:sperimentazione-tipo1-15nodi}
\end{table}

\subsection{Tipologia II - Grafi Random - 15 nodi [$n=15$] e probabilità $0.70$ [$p=0.70$]}

\begin{table}[H]
\centering
\begin{tabular}{|c|c|c|c|c|c|c|}
\hline
\textbf{k} & \textbf{a} & \textbf{n\_step} & \textbf{n\_usw} & \textbf{n\_esw} & \textbf{t} \\ \hline
2 & 82 & 7 & 5929 & 267 & 1 \\ \hline
3 & 82 & 14 & 7648 & 340 & 1 \\ \hline
4 & 82 & 15 & 8677 & 375 & 1 \\ \hline
5 & 82 & 21 & 8742 & 394 & 1 \\ \hline
6 & 82 & 21 & 9072 & 414 & 1 \\ \hline
7 & 82 & 22 & 9219 & 402 & 1 \\ \hline
8 & 82 & 11 & 9093 & 398 & 1 \\ \hline
9 & 82 & 16 & 9340 & 420 & 1 \\ \hline
10 & 82 & 21 & 9361 & 430 & 1 \\ \hline
11 & 82 & 18 & 9475 & 444 & 1 \\ \hline
12 & 82 & 19 & 9510 & 413 & 1 \\ \hline
13 & 82 & 14 & 9561 & 436 & 1 \\ \hline
14 & 82 & 20 & 9631 & 444 & 1 \\ \hline
15 & 82 & 19 & 9581 & 443 & 1 \\ \hline
\end{tabular}
\caption{Tipologia II - Grafi Random - 15 nodi e probabilità $0.70$}
\label{tab:sperimentazione-tipo1-15nodi}
\end{table}

\subsection{Tipologia II - Grafi Random - 30 nodi [$n=30$]}

\begin{table}[H]
\centering
\begin{tabular}{|c|c|c|c|c|c|c|}
\hline
\textbf{k} & \textbf{p} & \textbf{a} & \textbf{n\_step} & \textbf{n\_usw} & \textbf{n\_esw} & \textbf{t} \\ \hline
29 & 0.09 & 48 & 30 & 7507 & 117 & 2 \\ \hline
3 & 0.86 & 366 & 40 & 32140 & 808 & 2 \\ \hline
11 & 0.15 & 72 & 36 & 9206 & 179 & 2 \\ \hline
16 & 0.03 & 12 & 14 & 3207 & 106 & 2 \\ \hline
8 & 0.15 & 70 & 37 & 10159 & 89 & 2 \\ \hline
6 & 0.20 & 77 & 35 & 10730 & 193 & 2 \\ \hline
6 & 0.36 & 151 & 38 & 15663 & 296 & 2 \\ \hline
3 & 0.39 & 168 & 19 & 15748 & 286 & 2 \\ \hline
28 & 0.56 & 248 & 34 & 26688 & 633 & 2 \\ \hline
17 & 0.42 & 191 & 46 & 22487 & 376 & 2 \\ \hline
\end{tabular}
\caption{Tipologia II - Grafi Random - 30 nodi}
\label{tab:sperimentazione-tipo1-30nodi}
\end{table}

\subsection{Tipologia II - Grafi Random - 45 nodi [$n=45$]}

\begin{table}[H]
\centering
\begin{tabular}{|c|c|c|c|c|c|c|}
\hline
\textbf{k} & \textbf{p} & \textbf{a} & \textbf{n\_step} & \textbf{n\_usw} & \textbf{n\_esw} & \textbf{t} \\ \hline
11 & 0.69 & 666 & 74 & 70663 & 74 & 5 \\ \hline
19 & 0.82 & 810 & 56 & 86036 & 1522 & 5 \\ \hline
41 & 0.28 & 284 & 55 & 32495 & 351 & 5 \\ \hline
42 & 0.17 & 185 & 48 & 23126 & 225 & 5 \\ \hline
23 & 0.62 & 633 & 70 & 67407 & 953 & 5 \\ \hline
5 & 0.97 & 964 & 58 & 85815 & 1488 & 5 \\ \hline
25 & 0.23 & 205 & 48 & 25993 & 231 & 5 \\ \hline
29 & 0.58 & 557 & 57 & 58736 & 923 & 5 \\ \hline
9 & 0.49 & 481 & 62 & 51387 & 561 & 5 \\ \hline
15 & 0.32 & 338 & 52 & 37617 & 544 & 5 \\ \hline
\end{tabular}
\caption{Tipologia II - Grafi Random - 45 nodi}
\label{tab:sperimentazione-tipo1-45nodi}
\end{table}

\subsection{Tipologia II - Grafi Random - 60 nodi [$n=60$]}

\begin{table}[H]
\centering
\begin{tabular}{|c|c|c|c|c|c|c|}
\hline
\textbf{k} & \textbf{p} & \textbf{a} & \textbf{n\_step} & \textbf{n\_usw} & \textbf{n\_esw} & \textbf{t} \\ \hline
30 & 0.27 & 456 & 67 & 51948 & 514 & 10 \\ \hline
52 & 0.05 & 52 & 56 & 13850 & 101 & 10 \\ \hline
13 & 0.44 & 783 & 83 & 85091 & 956 & 10 \\ \hline
51 & 0.84 & 1484 & 87 & 158991 & 2163 & 10 \\ \hline
38 & 0.77 & 1331 & 107 & 139927 & 1832 & 10 \\ \hline
11 & 0.90 & 1584 & 94 & 161166 & 2144 & 10 \\ \hline
32 & 0.66 & 1201 & 98 & 124281 & 1336 & 10 \\ \hline
3 & 0.23 & 415 & 65 & 40080 & 374 & 10 \\ \hline
14 & 0.39 & 711 & 92 & 76014 & 609 & 10 \\ \hline
47 & 0.86 & 1507 & 90 & 158109 & 2236 & 10 \\ \hline
\end{tabular}
\caption{Tipologia II - Grafi Random - 60 nodi}
\label{tab:sperimentazione-tipo1-60nodi}
\end{table}

\subsection{Tipologia II - Grafi Random - 75 nodi [$n=75$]}

\begin{table}[H]
\centering
\begin{tabular}{|c|c|c|c|c|c|c|}
\hline
\textbf{k} & \textbf{p} & \textbf{a} & \textbf{n\_step} & \textbf{n\_usw} & \textbf{n\_esw} & \textbf{t} \\ \hline
14 & 0.04 & 107 & 80 & 17362 & 94 & 10 \\ \hline
24 & 0.89 & 2498 & 141 & 254295 & 2593 & 10 \\ \hline
23 & 0.23 & 703 & 99 & 79253 & 586 & 10 \\ \hline
2 & 0.35 & 960 & 44 & 64429 & 573 & 10 \\ \hline
41 & 0.07 & 207 & 74 & 28634 & 174 & 10 \\ \hline
10 & 0.33 & 918 & 95 & 99490 & 665 & 10 \\ \hline
73 & 0.10 & 310 & 77 & 39623 & 175 & 10 \\ \hline
16 & 0.14 & 400 & 96 & 46173 & 221 & 10 \\ \hline
8 & 0.08 & 221 & 85 & 28302 & 92 & 10 \\ \hline
6 & 0.29 & 803 & 115 & 85384 & 636 & 10 \\ \hline
\end{tabular}
\caption{Tipologia II - Grafi Random - 75 nodi}
\label{tab:sperimentazione-tipo1-75nodi}
\end{table}

\subsection{Tipologia II - Grafi Random - 90 nodi [$n=90$]}

\begin{table}[H]
\centering
\begin{tabular}{|c|c|c|c|c|c|c|}
\hline
\textbf{k} & \textbf{p} & \textbf{a} & \textbf{n\_step} & \textbf{n\_usw} & \textbf{n\_esw} & \textbf{t} \\ \hline
27 & 0.63 & 2554 & 173 & 265205 & 2301 & 15 \\ \hline
53 & 0.36 & 1468 & 144 & 158319 & 1059 & 15 \\ \hline
25 & 0.36 & 1466 & 132 & 156640 & 896 & 15 \\ \hline
35 & 0.18 & 722 & 109 & 81552 & 443 & 15 \\ \hline
45 & 0.47 & 1846 & 122 & 196421 & 1568 & 15 \\ \hline
20 & 0.17 & 652 & 122 & 72931 & 375 & 15 \\ \hline
7 & 0.35 & 1443 & 135 & 148117 & 977 & 15 \\ \hline
30 & 0.85 & 3433 & 154 & 357717 & 3387 & 15 \\ \hline
45 & 0.18 & 716 & 105 & 79042 & 379 & 15 \\ \hline
11 & 0.56 & 2230 & 156 & 227861 & 1881 & 15 \\ \hline
\end{tabular}
\caption{Tipologia II - Grafi Random - 90 nodi}
\label{tab:sperimentazione-tipo1-90nodi}
\end{table}

\subsection{Tipologia II - Grafi Random - 100 nodi [$n=100$]}

\begin{table}[H]
\centering
\begin{tabular}{|c|c|c|c|c|c|c|}
\hline
\textbf{k} & \textbf{p} & \textbf{a} & \textbf{n\_step} & \textbf{n\_usw} & \textbf{n\_esw} & \textbf{t} \\ \hline
13 & 0.64 & 3252 & 199 & 330952 & 2706 & 20 \\ \hline
3 & 0.56 & 2724 & 147 & 208293 & 1490 & 20 \\ \hline
12 & 0.38 & 1898 & 173 & 197154 & 1210 & 20 \\ \hline
15 & 0.28 & 1368 & 154 & 147745 & 873 & 20 \\ \hline
14 & 0.05 & 241 & 104 & 33638 & 100 & 20 \\ \hline
12 & 0.20 & 954 & 149 & 107103 & 522 & 20 \\ \hline
13 & 0.77 & 3790 & 234 & 381704 & 2963 & 20 \\ \hline
33 & 0.43 & 2119 & 161 & 225302 & 1483 & 20 \\ \hline
2 & 0.34 & 1688 & 71 & 112251 & 635 & 20 \\ \hline
25 & 0.01 & 48 & 62 & 10979 & 98 & 20 \\ \hline
\end{tabular}
\caption{Tipologia II - Grafi Random - 100 nodi}
\label{tab:sperimentazione-tipo1-100nodi}
\end{table}

\section{Conclusioni}
\justify
La suddetta sperimentazione effettuata ha pienamente soddisfatto le aspettative e gli obiettivi prefissati in fase di progettazione.\\
Sia la tipologia di sperimentazione I che la tipologia di sperimentazione II hanno prodotto e portato alla luce risultati significativi che evidenziano la validità e la correttezza dei teoremi affrontati nel Capitolo 2.\\
Inoltre l'assoluta accuratezza e precisione delle procedure applicate è testimoniata dall'esattezza degli output ottenuti.\\

Per ciò che concerne la Tipologia I, specifichiamo che l'enorme dispendio temporale e l'ingente quantità di risorse richieste dai calcoli effettuati, hanno limitato, nella pratica, l'attività di sperimentazione.\\
La quantità di risultati ottenuti è però sufficiente per trarre le dovute riflessioni e conclusioni.\\
Per completezza specifichiamo che il numero delle permutazioni (colorazioni) analizzate, per ciascun calcolo dell'ottimo, è dell'ordine del milione.\\
Si è scelto di restare all'interno di questa soglia in modo da ottenere esecuzioni singole che non superino le 1-2 ore al massimo.\\

Analizzando gli output ottenuti relativi alla sperimentazione di tipo I, possiamo affermare con certezza che molti parametri influiscono in modo diretto con le operazioni di calcolo dell'ottimo sia con funzione di benessere sociale utilitario che egalitario.\\
In particolare possiamo tralasciare l'analisi della complessità riguardante l'algoritmo per il calcolo della colorazione stabile poiché la grandezza dei grafi utilizzati determina un dispendio temporale trascurabile.\\

Concentrandoci invece sugli algoritmi per il calcolo degli ottimi con funzioni di benessere sociale utilitario e egalitario, possiamo affermare che il parametro $k$ influisce direttamente sulla complessità temporale relativa alle varie esecuzioni poiché determina il numero di colorazioni (permutazioni) da iterare.\\

Nello specifico, dato come valore fisso il numero di nodi $n$, l'uso del parametro $k$ in correlazione con quest'ultimo genera un numero di permutazioni pari a \(k^n\), è dunque immediato comprendere come al crescere di $k$ crescerà anche il numero di permutazioni \(k^n\) e di conseguenza la complessità temporale dell'algoritmo in modo esponenziale.\\

Un altro parametro che influenza direttamente la complessità temporale relativa agli algoritmi per il calcolo degli ottimi è $p$.\\
Il parametro $p$ definisce la probabilità di costruire archi tra le coppie di nodi, dunque determina il grado di densità del grafo in oggetto.\\
La forte connessione del grafo e dunque un valore di densità elevato per quest'ultimo, determinano una maggiore complessità temporale per gli algoritmi relativi al calcolo degli ottimi.\\
La presenza di molti archi nel grafo genera l'esistenza di un numero maggiore di nodi adiacenti per ciascun nodo del grafo e dunque aumenta in modo diretto il numero di iterazioni innestate all'interno dell'algoritmo.\\

Inizializzando con valori piccoli e medi i suddetti parametri $k$ e $p$ otteniamo esecuzioni caratterizzate da una complessità temporale minore.\\

Analizzando in seguito i parametri ottenuti, in particolare riguardo il prezzo dell'anarchia sperimentale utilitario e quello egalitario, possiamo notare dai risultati ottenuti come sia facile, in caso di grafi piccoli, ottenere colorazioni stabili con un benessere sociale utilitario e egalitario molto vicino se non pari al benessere sociale utilitario e egalitario delle colorazioni ottime.\\

Molto spesso infatti otteniamo il valore $1$ per ciò che riguarda il prezzo dell'anarchia sperimentale utilitario o egalitario o entrambi.\\
Tale valore conferma che la colorazione stabile trovata dall'algoritmo presenta un valore di benessere sociale utilitario o egalitario o entrambi pari al valore dei rispettivi ottimi.\\

In generale possiamo confermare che la totalità dei risultati ottenuti rispetta pienamente la descrizione del modello matematico e le affermazioni determinate dai vari teoremi presentati all'interno del Capitolo 2.\\
Possiamo affermare che il gioco implementato è convergente poiché sono stati sempre ottenuti risultati stabili (equilibri di Nash).\\
Abbiamo inoltre ottenuto valori per il prezzo dell'anarchia sperimentale utilitario e egalitario che rispettano a pieno le asserzioni contenute nel Teorema 1, poiché abbiamo ottenuto sempre risultati \(\leq 2\).\\

Per ciò che riguarda la sperimentazione di tipo II, l'analisi è meno complessa.\\
Anche qui il parametro $k$ influenza la computazione, sia al livello temporale che concettuale.\\
Il parametro $k$ influenza la complessità temporale dell'algoritmo in misura del tutto minore se confrontata con quella relativa agli algoritmi per il calcolo delle 2 tipologie di ottimo implementate per il gioco in oggetto.\\
Nonostante ciò il parametro determina l'aumento delle iterazioni innestate che interessano la ricerca del miglioramento per ciascun nodo, tale aumento è ovviamente direttamente proporzionale al crescere di $k$.\\

Il parametro $k$ influenza anche la complessità concettuale del gioco poiché ciascun giocatore, avendo più strategie, dovrà cercare più a lungo le mosse migliori, in modo da trovare la dinamica più adatta all'istanza corrente.\\

Il parametro $p$, anche in questa tipologia di sperimentazione, influenza direttamente la complessità temporale del calcolo poiché, aumentando la presenza di connessioni e dunque la densità del grafo corrente, aumenta anche il numero di nodi adiacenti per ciascun nodo e di conseguenza il numero totale del iterazioni necessarie.\\

In generale, per entrambe le tipologie di sperimentazione, l'aumento del valore del parametro $p$, che descrive la densità del grafo, produce un generale e banale aumento del valore del parametro $a$, il numero di archi dell'istanza corrente.\\

Il limitatore temporale $t$, in questo caso, serve a specificare un'altra caratteristica fondamentale di questo tipo di analisi effettuata sul gioco in oggetto, ovvero l'aumento graduale del tempo necessario a ciascuna esecuzione all'aumentare del numero di nodi $n$ (fissato per ogni sperimentazione).\\
Difficilmente il variare dei parametri $k$ e $p$ produce oscillazioni significative al livello temporale (come ad esempio un aumento netto del tempo richiesto) e infatti per la quasi totalità dei casi è corretto impostare un set di valori omogeneo per $t$.\\

Per concludere affermiamo inoltre che l'aumento del numero di nodi $n$, del numero dei colori $k$ e della probabilità che descrive la densità del grafo $p$, influenzano direttamente il parametro $n\_step$ generando nella quasi totalità dei casi un generale aumento più o meno significativo di quest'ultimo.\\
Inoltre possiamo specificare che al crescere di $n$, per entrambe le tipologie di sperimentazione, è possibile notare un generale aumento dei valori associati ai parametri relativi al benessere sociale utilitario, poiché abbiamo più membri all'interno della sommatoria definita dalla funzione.\\
Per ciò che riguarda i parametri legati al benessere sociale egalitario il discorso è leggermente differente, questi ultimi infatti sono molto influenzati dal valore di $p$ poiché un grafo più denso genera l'aumento generale del profitto individuale per ciascun giocatore e di conseguenza del profitto minimo individuale.\\ 

	
	\newpage
	\nocite{*}
	\printbibliography
	
	\chapter*{Ringraziamenti}
\justify
\blindtext
	
\end{document}
