\begin{flushleft}
\justify

\chapter{Equilibri di Nash}

\section{Teoria dei giochi}
La teoria dei giochi è la disciplina scientifica che si occupa dello studio e dell'analisi del comportamento e delle decisioni di soggetti razionali in un contesto di interdipendenza strategica.\\
Si definisce interdipendenza strategica, o interazione strategica, lo scenario in cui le decisioni di un individuo influenzano anche le scelte e gli scenari relativi agli altri individui.\\
Il principale oggetto di studio della teoria dei giochi sono le situazioni di conflitto nelle quali gli attori sono costretti ad intraprendere una strategia di competizione o cooperazione.\\
Tale scenario è definito gioco strategico e gli individui sono denominati giocatori.\\
Sulla base delle premesse e delle regole compositive del gioco in oggetto, viene costruito un modello matematico nel quale ciascun giocatore effettua le proprie decisioni (mosse migliorative) seguendo una strategia finalizzata ad aumentare il proprio vantaggio netto.\\
A ciascuna scelta positiva corrisponde un ritorno favorevole in termini di beneficio (payoff), il medesimo concetto vale in modo contrario in caso di scelta negativa, in tal caso il ritorno sarà sfavorevole.\\
In tali scenari le decisioni di un soggetto possono influire direttamente sui risultati conseguibili dagli altri e viceversa secondo un meccanismo di retroazione.\\
La teoria dei giochi è un concetto di soluzione applicabile ad un'ingente molteplicità di casi nei quali una pluralità di agenti decisionali possono operare in maniere competitiva, seguendo interessi contrastanti, o in maniera  cooperativa, seguendo l'interesse comune.\newline

\subsection{Giochi non-cooperativi}
In questo documento, la trattazione sarà incentrata sull'analisi di una particolare tipologia di giochi : i giochi non-cooperativi.\\
I giochi non-cooperativi, detti anche competitivi, rappresentano una specifica classe di giochi nella quale i giocatori non possono stipulare accordi vincolanti di cooperazione (anche normativamente), indipendentemente dai loro obiettivi.\\
Il criterio di comportamento razionale adottato nei giochi non-cooperativi è di carattere individuale ed è denominato strategia del massimo.\\
La suddetta definizione di razionalità va a modellare il comportamento di un individuo intelligente e ottimista che si prefigge l'obiettivo di prendere sempre la decisione che consegue il massimo guadagno possibile, perseguendo di conseguenza sempre la strategia più vantaggiosa per se stesso.\\
Si parla dunque di punto di equilibrio qualora nel gioco esista una strategia che presenti il massimo guadagno per tutti i giocatori, ovvero uno stato stabile del gioco nel quale tutti gli attori ottengono il massimo profitto individuale e collettivo.\newline

\section{Equilibri di Nash}
La precedente affermazione muove l'oggetto della trattazione verso l'argomento centrale di questo studio, ovvero gli equilibri di Nash.\\
L'equilibrio di Nash è una combinazione di strategie nella quale ciascun giocatore effettua la migliore scelta possibile, seguendo cioè una strategia dominante, sulla base delle aspettative di scelta dell'altro giocatore.\\
L'equilibrio di Nash è la combinazione di mosse (m1, m2) in cui la mossa di ciascun giocatore è la migliore risposta alla mossa effettuata da un'altro giocatore.\\
Ciascun giocatore formula delle aspettative sulla scelta dell'altro giocatore e in base a queste decide la propria strategia, con l'obiettivo di massimizzare il proprio profitto e di conseguenza quello degli altri.
Un equillibrio di Nash è un equilibrio stabile, poichè nessun giocatore ha interesse a modificare la propria strategia.\\
Ciascun giocatore trae la massima utilità possibile dalle proprie scelte, tenendo conto della migliore scelta dell'altro giocatore, e dunque qualunque variazione alla propria strategia potrebbe soltanto peggiorare il proprio valore di tornaconto (payoff o utilità).\\
L'equilibrio di Nash è conosciuto anche con il nome di equilibrio non cooperativo poichè rappresenta una situazione di equilibrio ottimale per un gioco non-cooperativo.\\
L'equilibrio di Nash non deriva dall'accordo tra i giocatori, bensì dall'adozione di strategie dominanti perseguite da tutti i giocatori, tali da garantire sia il migior profitto possibile per ciascun giocatore (ottimo individuale), sia il miglior equilibrio collettivo (ottimo sociale).\newline

\subsection{Definizione formale}
Definiamo ora alcuni concetti basilari e chiariamo alcuni aspetti matematici della teoria dei giochi al fine di delineare in modo più accurato il concetto di equilibrio di Nash.\\
Un gioco è caratterizzato da :
\begin{itemize}
	\item un insieme G di giocatori, o agenti, che indicheremo con \(i = 1,\ldots,N\)
	\item un insieme S di strategie, costituito da un insieme di M vettori \[S_{i}=\left(s_{{i,1}},s_{{i,2}},\ldots,s_{{i,j}},\ldots,s_{{i,M_{i}}}\right)\] ciasuno dei quali contiene l'insieme l'insieme delle strategie che il giocatore i-esimo ha a disposizione, cioè l'insieme delle azioni che esso può compiere.\\(indichiamo con \(s_i\) la strategia scelta dal giocatore \(i\))
	\item un insieme U di funzioni \[u_{i}=U_{i}\left(s_{1},s_{2},\ldots,s_{i},\ldots,s_{N}\right)\] che associano ad ogni giocatore \(i\) il guadagno (detto anche payoff) \(u_i\) derivamente da una data combinazione di strategie (il guadagno di un giocatore in generale non dipende solo dalla propria strategia ma anche dalle strategie scelte dagli avversari)
\end{itemize}
Un equilibrio di Nash per un dato gioco è una combinazione di strategie (che indichiamo con l'apice \(e\))
\[s_{1}^{e},s_{2}^{e},...,s_{N}^{e}\]
tale che
\[U_{i}\left(s_{1}^{e},s_{2}^{e},...,s_{i}^{e},...,s_{N}^{e}\right)\geq U_{i}\left(s_{1}^{e},s_{2}^{e},...,s_{i},...,s_{N}^{e}\right)\]
\(\forall i\) e \(\forall s_i\) scelta dal giocatore i-esimo.\newline

Il significato di quest'ultima disuguaglianza è il seguente : se un gioco ammette almeno un equilibrio di Nash, ciascun agente ha a dispozione almeno una strategia \(s_i^e\) dalla quale non ha alcun interesse ad allontanarsi se tutti gli altri giocatori hanno giocato la propria strategia \(s_j^e\).\\
Come si può facilmente desumere direttamente dalla suddetta disequazione, se il giocatore i gioca una qualunque strategia a sua dispozione diversa da \(s_i^e\), mentre tutti gli altri giocatori hanno giocato la propria strategia \(s_j^e\), può solo peggiorare il proprio guadagno o, al più, lasciarlo invariato.\\
Da qui si può dedurre quindi che se i giocatori raggiungono un eequilibrio di Nash, nessuno può più migliorare il proprio risultato modificando solo la propria strategia, ed è quindi vincolato alle scelte degli altri.\\
Poiché questo vale per tutti i giocatori, è evidente che se esiste un equilibrio di Nash ed è unico, esso rappresenta la soluzione del gioco, in quanto nessuno dei giocatori ha interesse a cambiare strategia.\newline

\begin{center}
  \begin{tabular}{ | l | c | r | }
    \cline{2-3}
    \multicolumn{1}{ c| }{} & \textbf{4} & \textbf{4} \\ \hline
    \textbf{4} & 5 & 6 \\ \hline
    \textbf{4} & 8 & 9 \\
    \hline
  \end{tabular}
\end{center}


\end{flushleft}