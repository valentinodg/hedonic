\chapter{Gioco della k-colorazione generalizzata}
\justify
Esaminiamo ora gli equilibri di Nash puri per il gioco della k-colorazione generalizzata nel quale viene fornito un grafo orientato e un insieme di k colori.\\
I nodi rappresentano i giocatori e gli archi catturano i loro reciproci interessi.\\
La strategia di ciascun giocatore è composta da k colori.\\
L'utilità di un giocatore v in un dato stato o colorazione è data dalla somma dei pesi degli archi \((v, u)\) incidenti a v tale che il colore scelto da v sia diverso da quello scelto da u, più il profitto guadagnato dall'utilizzo del colore scelto.\\
Per prima cosa dimostriamo che il gioco della k-colorazione generalizzata è convergente e dunque esiste sempre almeno un equilibrio di Nash per ogni istanza del gioco in questione.\\
Valutiamo dunque in seguito una descrizione delle prestazioni dei giochi della k-colorazione generalizzata per mezzo delle nozioni largamente utilizzate di prezzo dell'anarchia (price of anarchy) e prezzo della stabilità (price of stability).\\
Forniamo inoltre limiti stretti per 2 tipi di benessere sociale ampiamente utilizzati, il benessere sociale utilitario (utilitarian social welfare) e il benessere sociale egalitario (egalitarian social welfare).\newline

\section{Descrizione generale}
\justify
Le istanze appartenenti al gioco della k-colorazione generalizzata sono giocati su grafi non-orientati pesati in cui i nodi corrispondono ai giocati e in cui gli archi identificano le connessioni sociali o le relazioni tra i giocatori.\\
Il set di strategie di ciascun giocatore è un insieme di k colori disponibili (assumiamo che i colori siano gli stessi per ogni giocatore).\\
Quando i giocatori selezionano un colore inducono una colorazione k (o semplicemente una colorazione).\\
Ciascun giocatore possiede una funzione di profitto legata all'apprezzamento da parte di quest'ultimo del colore scelto (vale per tutti i colori disponibili per il giocatore).\\
Data una colorazione, l'utilità (o il guadagno) di un giocatore v colorato con il colore i è la somma dei pesi degli archi \((v, u)\) incidenti a v, tale che il colore scelto da v è diverso da quello scelto da u, più il profitto derivante dalla scelta del colore i da parte del giocatore v.\\
Assumiamo che i giocatori siano egoisti, dunque un concetto di soluzione ben noto per questo tipo di impostazione è l'equilibrio di Nash.\\
L'equilibrio di Nash è uno dei concetti più importanti nella teoria dei giochi e fornisce una soluzione stabile che è robusta alle deviazioni dei singoli giocatori.\\
Formalmente, una colorazione è un equilibrio di Nash puro se nessun giocatore può migliorare la propria utilità deviando unilateralmente dalla propria strategia attuale.\\
L'egoismo dei giocatori può causare in molti casi la perdita di benessere sociale e quindi una soluzione stabile non è sempre buona rispetto al benessere della società.\newline

Consideriamo ora 2 nozioni di benessere sociale, naturali e ampiamente utilizzate.\\
Data una colorazione k, il benessere sociale utilitario (utilitarian social welfare) è definito come la somma delle utilità dei giocatori nella colorazione k, mentre il benessere sociale egalitario (egalitarian sociale welfare) è definito come l'utilità minima tra tutti i giocatori nella colorazione k.\newline

Utilizziamo inoltre 2 metodi per misurare la bontà di un equilibrio di Nash rispetto a un benessere sociale, il prezzo dell'anarchia (price of anarchy) e il prezzo della stabilità (price of stability).\\
Il prezzo dell'anarchia descrive, nel peggiore dei casi, come l'efficienza di un sistema degrada a causa del comportamento egoistico dei suoi giocatori, mentre il prezzo della stabilità ha un naturale significato di stabilità, poiché è la soluzione ottimale tra quelle che possono essere accettate da giocatori egoisti.\\
Studiamo ora l'esistenza e le performance degli equilibri di Nash nei giochi della k-colorazione generalizzata.\\
Ci concentriamo solo sui grafi non-orientati poiché per i grafi orientati anche il problema di decidere se un'istanza ammetta un equilibrio di Nash è un problema difficile (NP-Hard), inoltre esistono casi per i quali un equilibrio di Nash non esiste affatto.\newline

\subsection{Nozioni sul problema}
\justify
Sappiamo che in caso di grafi non-orientati non-pesati è possibile calcolare un equilibrio di Nash in tempo polinomiale.\\
Nel nostro caso, il problema di calcolare un equilibrio di Nash su grafi non-orientati pesati è PLS-Completo anche per \(k = 2\), dato che il gioco del taglio massimo (Max-cut game) è un caso speciale del nostro gioco.\\
Proprio riguardo questo aspetto è bene delineare la relazione che esiste tra il gioco della k-colorazione generalizzata e il gioco del taglio massimo, un problema molto importante e ampiamente trattato in letteratura.\\
Il gioco della k-colorazione generalizzata è un estensione del gioco del taglio massimo, infatti quest'ultimo può essere ottenuto ponendo a 0 i profitti relativi ai colori e ponendo a 2 il numero di colori presenti nel set disponibile per ciascun giocatore.\\
Inoltre il gioco della k-colorazione generalizzata è un'estensione del gioco della k-colorazione nel quale vi sono k-colori ma i profitti relativi ai colori sono impostati a 0.\newline

\section{Dettagli sul modello}
\justify
Dato un grafo semplice non-orientato \(G = (V, E, w)\), dove \(|V| = n\), \(|E| = m\) e \(w : E\rightarrow\mathds{R}_{\geq 0}\) è la funzione per i pesi sugli archi che associa un peso positivo a ciascun arco.\\
Denotiamo con \(\delta^v (G) = \sum_{u \in V : \{v, u\} \in E} w(\{v, u\})\) la somma dei pesi di tutti gli archi incidenti a v.\\
L'insieme dei nodi con cui un nodo v ha un arco in comune è chiamato insieme dei vicini di v (insieme dei nodi adiacenti a v).\\
Un'istanza di gioco della k-colorazione generalizzata è un tupla \((G, K, P)\). \(G = (V, E, w)\) è un grafo pesato non-orientato senza self loops, in cui ogni \(v \in V\) è un giocatore egoista.\\
K è un insieme di colori disponibili (assumiamo \(K \geq 0\)).\\
Il set di strategia di ciascun giocatore è dato dai k colori disponibili, ovvero i giocatori hanno lo stesso insieme di azioni.\\
Denotiamo con \(P : V \times K \rightarrow \mathds{R}_{\geq 0}\) la funzione di profitto del colore, che definisce quanto un giocatore apprezza un colore, ovvero se il giocatore v scegli di usare il colore i, allora guadagna \(P_v (i)\).\\
Per ciascun giocatore v, definiamo \(P_v^M\) come il massimo profitto che v può guadagnare da un colore, formalmente \(P_v^M = max_{i=1,\ldots,k} P_v (i)\).\\
Quando \(P_v (i) = 0 \forall v \in V\) e \(\forall i \in k\), si ha il caso in cui non vi sono profitti associati ai colori scelti, quindi possiamo riferirci a questo gioco come a un gioco della k-colorazione (graph k-coloring game).\\
Uno stato del gioco \(c = \{c_1,\ldots,c_n\}\) è una k-colorazione, o semplicemente una colorazione, dove \(c_v\) è il colore (cioè un numero \(1 \leq c_v \leq k\)) scelto dal giocatore v.\\
In una determinata colorazione c, il payoff (o l'utilità) di un giocatore v è la somma dei pesi degli archi \((v, u)\) incidenti a v, tale che il colore scelto da v è diverso da quello scelto da u, oltre al profitto ottenuto dall'utilizzo il colore scelto.\\
In modo formale, per una colorazione c, il payoff di un giocatore v è \(\mu_c (v) = \sum_{u \in V:\{v, u\} \in E \wedge c_v \neq c_u} w(\{v, u\}) + P_v(c_v)\).\\
Quando un arco \((v, u)\) fornisce utilità ai suoi endpoints in una colorazione c, cioè quando \(c_v \neq c_u\) diciamo che tale arco è corretto.\\
diciamo anche che un arco \((v, u)\) è monocromatico in una colorazione c quanto \(c_v = c_u\).\\
Sia \(c_{-v}, c_u^{\prime}\) la colorazione ottenuta da c cambiando la strategia del giocatore v da \(c_v\) a \(c_v^{\prime}\).\\
Data una colorazione \(c = \{c_1,\ldots,c_n\}\), una mossa migliorativa (improving move) del giocatore v nella colorazione c è una strategia \(c_v^{\prime}\) tale che \(\mu_{(c_{-v}, c_v^{\prime})} (v) > \mu_c (v)\).\\
Uno stato del gioco è un equilibrio di Nash puro o equilibrio stabile se e so se nessun giocatore può effettuare una mossa migliorativa.\\
In modo formale, \(c = \{c_1,\ldots,c_n\}\) è un equilibrio di Nash se \(\mu_c (v) \geq \mu_{(c_{-v}, c_v^{\prime})} (v)\) per ogni possibile colorazione \(c_v^{\prime}\) e per ogni giocatore \(v \in V\).\\
Una dinamica di miglioramento (improving dynamic), o brevemente dinamica (dynamic), è una sequenza di mosse migliorative. Si dice che un gioco sia convergente se, dato un qualsiasi stato iniziale c, qualsiasi sequenza di mosse migliorative porta a un equilibrio di Nash.\newline
Data una colorazione c, definiamo una funzione di benessere sociale utilitario (utilitarian social welfare) \(SW_{UT}(c)\) e una funzione di benessere sociale egalitario (egalitarian sociale welfare) \(SW_{EG}(c)\) come segue :
\[SW_{UT} (c) = \sum_{v \in V} \mu_c (v) = \sum_{v \in V} P_v(c_v) + \sum_{\{v, u\} \in E : c_v \neq c_u} 2w(\{v, u\})\]
\[SW_{EG} (c) = min_{v \in V} \mu_c (v)\]
Indichiamo con C l'insieme di tutte le possibili colorazioni e denotiamo con Q l'insieme di tutte le colorazioni stabili. Data una funzione di benessere sociale SW, definiamo il prezzo dell'anarchia (price of anarchy) (PoA) per il gioco della k-colorazione generalizzata come il rapporto tra il massimo benessere sociale tra tutte le possibili colorazioni sul minimo benessere sociale tra tutte le possibili colorazioni stabili.\\
In modo formale, \(PoA = \frac{max_{c \in C} SW(c)}{min_{c^{\prime} \in Q} SW(c^{\prime})}\).\\
Definiamo inoltre il prezzo della stabilità (price of stability) (PoS) per il gioco della k-colorazione generalizzata come il rapporto tra il massimo benessere sociale tra tutte le possibili colorazioni sul massimo benessere sociale tra tutte le possibili colorazioni stabili.\\
In modo formale, \(PoS = \frac{max_{c \in C} SW(c)}{max_{c^{\prime} \in Q} SW(c^{\prime})}\).\\
Intuitivamente, il PoA (rispettivamente PoS) ci dice quanto è peggiore il benessere sociale nel peggiore (rispettivamente migliore) equilibrio di Nash, relativo al benessere sociale dell'ottimo.\\

\subsection{Convergenza ed esistenza degli equilibri di Nash}
\justify
Per prima cosa mostriamo che il gioco della k-colorazione generalizzata è convergente. Ciò implica chiaramente che gli equilibri di Nash esistono sempre.\\

\begin{prop}
	\(\forall k\), ogni gioco della k-colorazione generalizzata finito \((G, K, P)\) è convergente
\end{prop}

Notiamo che, da un lato, se il grafo non è pesato, la dinamica, partendo dalla colorazione in cui ogni giocatore v selezione il colore in modo tale da ottenere il massimo profitto possibile, che è, il colore i tale che \(P_v (i) = P_v^M\), converge ad un equilibrio di Nash in al massimo \(|E|\) mosse migliorative.\\
D'altra parte, se il grafo è pesato, il calcolo di un equilibrio di Nash è PLS-Completo.\\
Ne consegue il fatto che, quando \(k = 2\), il nostro gioco è una generalizzazione del gioco del taglio (cut game) che è uno dei primi problemi che si sono dimostrati essere PLS-Completi.\newline

\section{Benessere sociale utilitario (utilitarian social welfare)}
\justify
In questa sezione ci concentreremo sul benessere sociale utilitario. Mostriamo limiti stretti per il prezzo dell'anarchia utilitario e tralasciamo quelli per il prezzo della stabilità utilitario.\newline

\subsection{Prezzo dell'anarchia utilitario (utilitarian price of anarchy)}
\justify
Ricordiamo che nel caso senza profitti associati ai colori, il prezzo dell'anarchia utilitario è esattamente \(\frac{k}{(k-1)}\).\\
Qui dimostriamo che il gioco della k-colorazione generalizzata il prezzo dell'anarchia utilitario è pari a 2, cioè indipendente dal numero di colori.\\
Iniziamo dimostrando che il prezzo dell'anarchia utilitario è al più 2.\\

\begin{theorem}
	Il prezzo dell'anarchia per il gioco della k-colorazione generalizzata è al più 2. 
\end{theorem}

Mostriamo ora che il prezzo dell'anarchia utilitario è almeno 2 anche per il caso speciale di grafi stella non-pesati.\\

\begin{theorem}
	Il prezzo dell'anarchia utilitario per il gioco della k-colorazione generalizzata è almeno 2, anche per il caso speciale di grafi stella non-pesati.
\end{theorem}

\section{(Benessere sociale egalitario (egalitarian social welfare))}
\justify
In questa sezione ci concentriamo sul benessere sociale egalitario. Mostriamo limiti stretti per il prezzo dell'anarchia egalitario e tralasciamo quelli per il prezzo della stabilità egalitario.\\

\subsection{Prezzo dell'anarchia egalitario (egalitarian price of anarchy)}
\justify