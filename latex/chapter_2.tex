\chapter{Gioco della k-colorazione generalizzata}
\justify
Esaminiamo ora gli equilibri di Nash puri per il gioco della k-colorazione generalizzata nel quale viene fornito un grafo orientato e un insieme di k colori.\\
I nodi rappresentano i giocatori e gli archi catturano i loro reciproci interessi.\\
La strategia di ciascun giocatore è composta da k colori.\\
L'utilità di un giocatore v in un dato stato o colorazione è data dalla somma dei pesi degli archi (v, u) incidenti a v tale che il colore scelto da v sia diverso da quello scelto da u, più il profitto guadagnato dall'utilizzo del colore scelto.\\
Per prima cosa dimostriamo che il gioco della k-colorazione generalizzata è convergente e dunque esiste sempre almeno un equilibrio di Nash per ogni istanza del gioco in questione.\\
Valutiamo dunque in seguito una descrizione delle prestazioni dei giochi della k-colorazione generalizzata per mezzo delle nozioni largamente utilizzate di prezzo dell'anarchia (price of anarchy) e prezzo della stabilità (price of stability).\\
Forniamo inoltre limiti stretti per 2 tipi di benessere sociale ampiamente utilizzati, il benessere sociale utilitario (utilitarian social welfare) e il benessere sociale egalitario (egalitarian social welfare).\newline

\section{Descrizione generale}
\justify
Le istanze appartenenti al gioco della k-colorazione generalizzata sono giocati su grafi non-orientati pesati in cui i nodi corrispondono ai giocati e in cui gli archi identificano le connessioni sociali o le relazioni tra i giocatori.\\
Il set di strategie di ciascun giocatore è un insieme di k colori disponibili (assumiamo che i colori siano gli stessi per ogni giocatore).\\
Quando i giocatori selezionano un colore inducono una colorazione k (o semplicemente una colorazione).\\
Ciascun giocatore possiede una funzione di profitto legata all'apprezzamento da parte di quest'ultimo del colore scelto (vale per tutti i colori disponibili per il giocatore).\\
Data una colorazione, l'utilità (o il guadagno) di un giocatore v colorato con il colore i è la somma dei pesi degli archi (v, u) incidenti a v, tale che il colore scelto da v è diverso da quello scelto da u, più il profitto derivante dalla scelta del colore i da parte del giocatore v.\\
Assumiamo che i giocatori siano egoisti, dunque un concetto di soluzione ben noto per questo tipo di impostazione è l'equilibrio di Nash.\\
L'equilibrio di Nash è uno dei concetti più importanti nella teoria dei giochi e fornisce una soluzione stabile che è robusta alle deviazioni dei singoli giocatori.\\
Formalmente, una colorazione è un equilibrio di Nash puro se nessun giocatore può migliorare la propria utilità deviando unilateralmente dalla proria strategia attuale.\\
L'egoismo dei giocatori può causare in molti casi la perdita di benessere sociale e quindi una soluzione stabile non è sempre buona rispetto al benessere della società.\newline

Consideriamo ora 2 nozioni di benessere sociale, naturali e ampiamente utilizzate.\\
Data una colorazione k, il benessere sociale utilitario (utilitarian social welfare) è definito come la somma delle utilità dei giocatori nella colorazione k, mentre il benessere sociale egalitario (egalitarian sociale welfare) è definito come l'utilità minima tra tutti i giocatori nella colorazione k.\newline

Utilizziamo inoltre 2 metodi per misurare la bontà di un equilibrio di Nash rispetto a un benessere sociale, il prezzo dell'anarchia (price of anarchy) e il prezzo della stabilità (price of stability).\\
Il prezzo dell'anarchia descrive, nel peggiore dei casi, come l'efficienza di un sistema degrada a causa del comportamento egoistico dei suoi giocatori, mentre il prezzo della stabilità ha un naturale significato di stabilità, poiché è la soluzione ottimale tra quelle che possono essere accettate da giocatori egoisti.\\
Studiamo ora l'esistenza e le performance degli equilibri di Nash nei giochi della k-colorazione generalizzata.\\
Ci concentriamo solo sui grafi non-orientati poiché per i grafi orientati anche il problema di decidere se un'istanza ammetta un equilibrio di Nash è un problema difficile (NP-Hard), inoltre esistono casi per i quali un equilibrio di Nash non esiste affatto.\newline

\subsection{Nozioni sul problema}
\justify
Sappiamo che in caso di grafi non-orientati non-pesati è possibile calcolare un equilibrio di Nash in tempo polinomiale.\\
Nel nostro caso, il problema di calcolare un equilibrio di Nash su grafi non-orientati pesati è PLS-Completo anche per \(k = 2\), dato che il gioco del taglio massimo (Max-cut game) è un caso speciale del nostro gioco.\\
Proprio riguardo questo aspetto è bene delineare la relazione che esiste tra il gioco della k-colorazione generalizzata e il gioco del taglio massimo, un problema molto importante e ampiamente trattato in letteratura.\\
Il gioco della k-colorazione generalizzata è un estensione del gioco del taglio massimo, infatti quest'ultimo può essere ottenuto ponendo a 0 i profitti relativi ai colori e ponendo a 2 il numero di colori presenti nel set disponibile per ciascun giocatore.\\
Inoltre il gioco della k-colorazione generalizzata è un'estensione del gioco della k-colorazione nel quale vi sono k-colori ma i profitti relativi ai colori sono impostati a 0.\newline

\section{Dettagli sul modello}
\justify
