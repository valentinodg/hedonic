%descrizione modulo generazione singola e multipla grafi randomici
%descrizione assunzioni (numero nodi, numero colori, numero grafi, ecc)
%descrizione 2 tipologie di sperimentazione nash+uoptnash+eoptnash+uopt+eopt+upoa+epoa e nash+uoptnash+eoptnash
%tabelle
%conclusioni


\chapter{Sperimentazione}

In questo capitolo esplicheremo e descriveremo gli aspetti e le caratteristiche generali relative all'attività di sperimentazione.\\

Inizieremo descrivendo in breve i moduli relativi alla generazione e alla lettura di grafi randomici situati all'interno dei 2 programmi implementati : generator.py e reader.py.\\

Procederemo delineando un elenco delle assunzioni e delle decisioni compiute che riguardano in generale la modellazione del problema e nello specifico la costruzione e la tipologia dei grafi implementati.\\
Queste rappresentano la base di partenza sulla quale è stata portata avanti l'attività di sperimentazione e sono fondamentali per comprendere meglio la parte concettuale e gli obiettivi dietro quest'ultima.\\

L'ultima fase prevede la presentazione delle conclusioni associate ai risultati ottenuti, in relazione al problema trattato (gioco della k-colorazione generalizzata) e in relazione alla trattazione matematica e ai teoremi delineati all'interno del documento alla base di questo studio (Generalized Graph k-Coloring Games).\\

\section{gnp\_random\_graph (grafo di Erdős-Rényi o grafo binomiale)}
\justify
Il principali oggetti matematici studiati durante l'attività di sperimentazione sono i grafi randomici.\\

In particolare è stata dedicata una porzione del codice all'interno del programma generator.py per la creazione e la manipolazione di una specifica tipologia di grafo : il gnp\_random\_graph.\\

Il gnp\_random\_graph, conosciuto anche come grafo di Erdős-Rényi o grafo binomiale, è l'oggetto attorno al quale ruota l'intera attività di implementazione.\\ Quest'ultimo appartiene alla classe Random Graphs.\\
Tale classe è contenuta all'interno della libreria di generazione e manipolazione grafi NetworkX utilizzata durante l'attività di implementazione, come specificato nella sezione precedente.\\
La tipologia di grafo gnp\_random\_graph, come delineato in precedenza, possiede dunque un costruttore di classe all'interno della suddetta libreria.\\

Il costruttore di classe, opportunamente parametrizzato in modo automatico oppure grazie all'intervento attivo dell'utente, è stato utilizzato per la creazione delle varie istanze di questa tipologia di grafo.\\

L'esecuzione dell'algoritmo di generazione comporta un complessità temporale pari a \(O(n^2)\), dunque un valore accettabile.\\
Si cita in questo senso la presenza di algoritmi di generazione più rapidi correlati a differenti costruttori, come ad esempio l'algoritmo relativo alla tipologia fast\_gnp\_random\_graph, che per valori piccoli di $p$ (probabilità di generare archi tra coppie di nodi) riesce a generare grafi sparsi in \(O(n+m)\).\\

Dato che la medio-alta densità dei grafi rende più interessante lo studio trattato è stato scelto di tralasciare questo tipo di costruttore alternativo e di utilizzare solo ed esclusivamente il costruttore gnp\_random\_graph per la generazione.\\

Descriviamo ora le modalità di generazione per le esecuzioni in modalità SINGLE EXEC e in modalità MULTIPLE EXEC e procediamo con la delineazione dell'assunzioni fatte durante la modellazione dei grafi creati.\\

\subsection{Generazione gnp\_random\_graph)}
\justify
Le funzioni per la generazione di grafi gnp\_random\_graph sono raggiungibili all'interno del programma generator.py selezionando la modalità di esecuzione MULTIPLE MODE e selezionando in seguito la classe gnp\_random\_graph.\\

All'utente è richiesto, come primo parametro, l'inserimento del numero di iterazioni da compiere, che corrispondono al numero di grafi da costruire all'interno del singolo processo di generazione corrente.\\

In seguito l'utente dovrà inserire il numero di nodi per i grafi da generare.\\
Tale valore, per assunzione, sarà un valore fisso per ciascun ciclo di generazione e dunque applicato a tutti i grafi appartenenti a quest'ultimo.\\

Gli ultimi parametri da inserire sono i valori massimo e minimo all'interno dei quali oscilleranno randomicamente i pesi associati agli archi del grafo, il tutto avviene nelle modalità specificate nella sezione relativa al generatore.\\

A questo punto l'utente sarà chiamato a scegliere la sotto modalità di generazione dei grafi, la scelta della modalità "single" comporterà la creazione di un path specifico all'interno della cartella gen nel quale verranno salvati i vari risultati della creazione seguendo lo schema presentato durante la descrizine della modalità SINGLE MODE relativa al generatore di grafi.\\
La scelta della modalità "multiple" comporterà la creazione di un path specifico all'interno della cartella mgen nel quale verranno salvati i vari risultati della creazione seguendo lo schema presentato durante la descrizine della modalità MULTIPLE MODE relativa al generatore di grafi. \\

continua ....\\

\section{Risultati sperimentazione I}
\justify
testo \\

\begin{table}[h]
\scalebox{0.9} {
\begin{tabular}{|c|c|c|c|c|c|c|c|c|c|c|}
\hline
\textbf{n} & \textbf{c} & \textbf{p} & \textbf{a} & \textbf{o\_usw} & \textbf{o\_esw} & \textbf{n\_count} & \textbf{n\_usw} & \textbf{n\_esw} & \textbf{u\_poa} & \textbf{e\_poa} \\ \hline
5 & 5 & 0.90 & 0 & 314 & 45 & 5 & 314 & 41 & 1 & 1.0975 \\ \hline
5 & 4 & 0.27 & 5 & 485 & 94 & 4 & 485 & 94 & 1 & 1 \\ \hline
5 & 3 & 0.82 & 9 & 410 & 45 & 4 & 410 & 40 & 1 & 1.1250 \\ \hline
 &  &  &  &  &  &  &  &  &  &  \\ \hline
\end{tabular}
}
\end{table}

\section{Risultati sperimentazione II}
\justify
testo \\