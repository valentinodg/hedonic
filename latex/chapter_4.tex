\chapter{Sperimentazione}
\justify
In questo capitolo esplicheremo e descriveremo gli aspetti e le caratteristiche generali relative all'attività di sperimentazione.\\

Inizieremo descrivendo in breve i moduli relativi alla generazione e alla lettura di grafi randomici situati all'interno dei 2 programmi implementati : generator.py e reader.py.\\

Procederemo delineando un elenco delle assunzioni e delle decisioni compiute che riguardano in generale la modellazione del problema e nello specifico la costruzione e la tipologia dei grafi implementati.\\
Queste rappresentano la base di partenza sulla quale è stata portata avanti l'attività di sperimentazione e sono fondamentali per comprendere meglio la parte concettuale e gli obiettivi dietro quest'ultima.\\

L'ultima fase prevede la presentazione delle conclusioni associate ai risultati ottenuti, in relazione al problema trattato (gioco della k-colorazione generalizzata) e in relazione alla trattazione matematica e ai teoremi delineati all'interno del documento alla base di questo studio (Generalized Graph k-Coloring Games).\\

\section{gnp\_random\_graph (grafo di Erdős-Rényi o grafo binomiale)}
\justify
Il principali oggetti matematici studiati durante l'attività di sperimentazione sono i grafi randomici.\\

In particolare è stata dedicata una porzione del codice all'interno del programma generator.py per la creazione e la manipolazione di una specifica tipologia di grafo : il gnp\_random\_graph.\\

Il gnp\_random\_graph, conosciuto anche come grafo di Erdős-Rényi o grafo binomiale, è l'oggetto attorno al quale ruota l'intera attività di implementazione.\\ Quest'ultimo appartiene alla classe Random Graphs.\\
Tale classe è contenuta all'interno della libreria di generazione e manipolazione grafi NetworkX utilizzata durante l'attività di implementazione, come specificato nella sezione precedente.\\
La tipologia di grafo gnp\_random\_graph, come delineato in precedenza, possiede dunque un costruttore di classe all'interno della suddetta libreria.\\

Il costruttore di classe, opportunamente parametrizzato in modo automatico oppure grazie all'intervento attivo dell'utente, è stato utilizzato per la creazione delle varie istanze di questa tipologia di grafo.\\

L'esecuzione dell'algoritmo di generazione comporta un complessità temporale pari a \(O(n^2)\), dunque un valore accettabile.\\
Si cita in questo senso la presenza di algoritmi di generazione più rapidi correlati a differenti costruttori, come ad esempio l'algoritmo relativo alla tipologia fast\_gnp\_random\_graph, che per valori piccoli di $p$ (probabilità di generare archi tra coppie di nodi) riesce a generare grafi sparsi in \(O(n+m)\).\\

Dato che la medio-alta densità dei grafi rende più interessante lo studio trattato è stato scelto di tralasciare questo tipo di costruttore alternativo e di utilizzare solo ed esclusivamente il costruttore gnp\_random\_graph per la generazione.\\

Descriviamo ora le modalità di generazione per le esecuzioni in modalità SINGLE EXEC e in modalità MULTIPLE EXEC e procediamo con la delineazione delle assunzioni formulate durante la modellazione dei grafi creati.\\

\subsection{Generazione gnp\_random\_graph)}
\justify
Le funzioni per la generazione di grafi gnp\_random\_graph sono raggiungibili all'interno del programma generator.py selezionando la modalità di esecuzione MULTIPLE MODE e selezionando in seguito la classe gnp\_random\_graph.\\

All'utente è richiesto, come primo parametro, l'inserimento del numero di iterazioni da compiere, che corrispondono al numero di grafi da costruire all'interno del singolo processo di generazione corrente.\\

In seguito l'utente dovrà inserire il numero di nodi per i grafi da generare.\\
Tale valore, per assunzione, sarà un valore fisso per ciascun ciclo di generazione e dunque applicato a tutti i grafi appartenenti a quest'ultimo.\\

Gli ultimi parametri da inserire sono i valori massimo e minimo all'interno dei quali oscilleranno randomicamente i pesi associati agli archi del grafo, il tutto avviene nelle modalità specificate nella sezione relativa al generatore.\\

A questo punto l'utente sarà chiamato a scegliere la sotto modalità di generazione dei grafi, la scelta della modalità "single" comporterà la creazione di un path specifico all'interno della cartella gen nel quale verranno salvati i vari risultati della creazione seguendo lo schema presentato durante la descrizione della modalità SINGLE MODE relativa al generatore di grafi.\\
La scelta della modalità "multiple" comporterà la creazione di un path specifico all'interno della cartella mgen nel quale verranno salvati i vari risultati della creazione seguendo lo schema presentato durante la descrizione della modalità MULTIPLE MODE relativa al generatore di grafi. \\

L'ultimo valore, gestito in modo randomico per ciascuna istanza della generazione corrente, è p.\\
Il parametro p rappresenta la probabilità di generare un archi tra le varie coppie di nodi del grafo corrente, viene fatto oscillare in modo randomico tra i valori \(0 \geq p \geq 1\).\\
La variabile p conterrà dunque un valore flottante (tipo float) la cui precisione è gestita in modo automatico e impostata a 2, ovvero sono ammesse fino a 2 cifre dopo la virgola.\\

Tralasciando i dettagli trascurabili descritti all'interno della sezione relativa all'implementazione, specifichiamo che la manipolazione del filesystem finalizzata al salvataggio degli output di generazione è effettuata in modo automatico e dinamico dal programma.\\
Quest'ultima è ottimizzata, cross-platform e presenta, come descritto in precedenza, primitive e funzioni appartenenti alla sola Standard Library del linguaggio Python.\\
Ovviamente presenta le medesimi caratteristiche dei moduli utilizzati e descritti in precedenza e dunque vengono accuratamente gestiti e evitati possibili conflitti tra i dati e problemi di inconsistenza.\\

\subsection{Assunzioni generali}
\justify
In breve, durante l'attività di progettazione e programmazione riguardanti la sperimentazione, sono state effettuate alcune decisioni relative a quest'ultima e sono state formulate le seguenti assunzioni sulla base delle quali si è svolta l'intera analisi.\\

\begin{itemize}
	\item I risultati delle varie sessioni di sperimentazione sono presentate al lettore sotto forma di tabelle riassuntive correlate da semplici leggende riguardanti la definizione dei parametri al fine di rendere più chiara e semplice la lettura e la visione dei dati raccolti
	\item Ogni sperimentazione si è svolta fissando il numero di nodi e facendo oscillare randomicamente gli altri parametri come ad esempio il numero di color $k$ oppure la probabilità relativa alla densità del grafo $p$.\\
	\item Il parametro $p$ relativo alla densità di un grafo, è un valore flottante a precisione 2 (2 cifre dopo la virgola) che oscilla tra i valori \(0 \geq p \geq 1\) e che descrive la probabilità di creare un arco tra un coppia di nodi e dunque definisce quanto il grafo in oggetto sia sparso (o denso)
	\item Il parametro $k$ definisce il numero di colori utilizzati per generare le differenti colorazioni, ciascuna delle quali rappresenta uno stato del gioco in oggetto. Ciascun colore \(i \in K\), con \(K = 1,\ldots,k\) (il set di colori), rappresenta una strategia per un giocatore. Per convenzione (la sperimentazione effettuata perderebbe di senso altrimenti) il parametro $k$ deve essere \(k \leq n\), con $n$ uguale al numero di nodi del grafo corrente
	\item Per ogni sessione di sperimentazione sono elencati 10 risultati rilevanti (quando possibile)
	\item Per le esecuzioni più pesati è stato concesso un limite temporale più ampio, ad esempio da 1 minuto a 2 minuti
	\item Il range all'interno del quale oscillano dinamicamente i pesi degli archi, per ciascuna esecuzione, è definito per convenzione \(0 \geq a \geq 100\), con $a$ uguale ai pesi degli archi
	\item Il range all'interno del quale oscillano i valore dei profitti associati ai colori per ciascun giocatore, per ciascuna esecuzione, è definito per convenzione \(0 \geq p \geq 100\), con $p$ uguale ai profitti associati ai colori per ogni giocatore
\end{itemize}

\subsection{Tipologie di sperimentazione}
\justify
Le tipologie di sperimentazione effettuate sono le seguenti

\begin{itemize}
	\item [\textbf{Tipologia I}] vengono eseguite in sequenza le seguenti operazioni su istanze di dimensione modesta, a causa dell'enorme dispendio temporale di alcune operazioni :
	\begin{enumerate}
		\item calcolo dell'ottimo relativo alla funzione di benessere sociale utilitario con limitatore temporale
		\item calcolo dell'ottimo relativo alla funzione di benessere sociale egalitario con limitatore temporale
		\item calcolo della colorazione stabile relativa alla definizione di equilibrio di Nash senza limitatore temporale
		\item definizione del valore di benessere sociale utilitario relativo alla colorazione stabile
		\item definizione del valore di benessere sociale egalitario relativo alla colorazione stabile
		\item definizione del valore relativo al prezzo dell'anarchia sperimentale utilitario
		\item definizione del valore relativo al prezzo dell'anarchia sperimentale egalitario
	\end{enumerate}
	\item [\textbf{Tipologia II}] vengono eseguite in sequenza le seguenti operazioni su istanze di dimensione medio-grande :
	\begin{enumerate}
		\item calcolo della colorazione stabile relativa alla definizione di equilibrio di Nash con limitatore temporale
		\item definizione del valore di benessere sociale utilitario relativo alla colorazione stabile
		\item definizione del valore di benessere sociale egalitario relativo alla colorazione stabile
	\end{enumerate}
\end{itemize}

\section{Risultati sperimentazione - Tipologia I}
\justify
Qui di seguito vengono presentati i risultati relativi alla sperimentazione di Tipologia I.\\

\textbf{LEGGENDA : }

\begin{itemize}
	\item Il parametro \textbf{k} descrive il numero di colori relativo all'istanza corrente
	\item Il parametro \textbf{p} descrive la probabilità relativa alla densità dell'istanza corrente
	\item Il parametro \textbf{a} descrive il numero di archi dell'istanza corrente
	\item Il valore \textbf{o\_usw} è il risultato ottenuto dall'applicazione, sull'istanza corrente, del calcolo dell'ottimo con funzione di benessere sociale utilitario
	\item Il valore \textbf{o\_esw} è il risultato ottenuto dall'applicazione, sull'istanza corrente, del calcolo dell'ottimo con funzione di benessere sociale egalitario
	\item Il valore \textbf{n\_step} è il numero di step (mosse migliorative effettuate) ottenuto dall'applicazione, sull'istanza corrente, del calcolo della colorazione stabile seguendo la definizione di equilibrio di Nash
	\item Il valore \textbf{n\_usw} è il risultato del calcolo del benessere sociale utilitario relativo alla colorazione stabile trovata per l'istanza corrente
	\item Il valore \textbf{n\_esw} è il risultato del calcolo del benessere sociale egalitario relativo alla colorazione stabile trovata per l'istanza corrente
	\item Il valore \textbf{u\_poa} è il risultato del calcolo del prezzo dell'anarchia utilitario sperimentale relativo all'istanza corrente
	\item Il valore \textbf{e\_poa} è il risultato del calcolo del prezzo dell'anarchia egalitario sperimentale relativo all'istanza corrente
	\item Il parametro \textbf{t} è valore, assegnato dinamicamente, del limitatore temporale (1 = 1 minuto, 2 = 2 minuti, ecc...), il valore vale per ciascun calcolo (assegnando \(t = 5\), l'esecuzione dovrebbe durare circa 10-15 minuti in totale)
\end{itemize}


\subsection{Sperimentazione - Tipologia I - Random 3 nodi}

\begin{table}[H]
\centering
\scalebox{0.9} {
\begin{tabular}{|c|c|c|c|c|c|c|c|c|c|c|}
\hline
\textbf{k} & \textbf{p} & \textbf{a} & \textbf{o\_usw} & \textbf{o\_esw} & \textbf{n\_step} & \textbf{n\_usw} & \textbf{n\_esw} & \textbf{u\_poa} & \textbf{e\_poa} & \textbf{t} \\ \hline
3 & 0.90 & 2 & 496 & 126 & 4 & 496 & 126 & 1 & 1 & 1 \\ \hline
2 & 0.55 & 3 & 406 & 120 & 1 & 406 & 91 & 1 & 1.3118 & 1 \\ \hline
3 & 0.71 & 2 & 486 & 91 & 2 & 486 & 91 & 1 & 1 & 1 \\ \hline
2 & 0.68 & 3 & 425 & 87 & 2 & 425 & 81 & 1 & 1.0740 & 1 \\ \hline
2 & 0.46 & 2 & 468 & 97 & 2 & 468 & 97 & 1 & 1 & 1 \\ \hline
\end{tabular}
}
\caption{Sperimentazione - Tipologia I - Random 3 nodi}
\label{tab:sperimentazione-tipo1-3nodi}
\end{table}


\subsection{Sperimentazione - Tipologia I - Random 5 nodi}

\begin{table}[H]
\centering
\scalebox{0.9} {
\begin{tabular}{|c|c|c|c|c|c|c|c|c|c|c|}
\hline
\textbf{k} & \textbf{p} & \textbf{a} & \textbf{o\_usw} & \textbf{o\_esw} & \textbf{n\_step} & \textbf{n\_usw} & \textbf{n\_esw} & \textbf{u\_poa} & \textbf{e\_poa} & \textbf{t} \\ \hline
5 & 0.90 & 9 & 314 & 45 & 5 & 314 & 41 & 1 & 1.0975 & 1 \\ \hline
4 & 0.27 & 5 & 485 & 94 & 4 & 485 & 94 & 1 & 1 & 1 \\ \hline
3 & 0.82 & 9 & 410 & 45 & 4 & 410 & 40 & 1 & 1.1250 & 1 \\ \hline
4 & 0.14 & 4 & 394 & 40 & 4 & 394 & 40 & 1 & 1 & 1 \\ \hline
2 & 0.79 & 7 & 420 & 37 & 5 & 420 & 37 & 1 & 1 & 1 \\ \hline
5 & 0.65 & 9 & 497 & 86 & 6 & 497 & 86 & 1 & 1 & 2 \\ \hline
3 & 0.40 & 6 & 805 & 120 & 5 & 766 & 120 & 1.0509 & 1 & 1 \\ \hline
3 & 0.52 & 5 & 932 & 85 & 4 & 932 & 85 & 1 & 1 & 1 \\ \hline
5 & 1 & 10 & 913 & 149 & 3 & 913 & 149 & 1 & 1 & 2 \\ \hline
2 & 0.24 & 3 & 590 & 64 & 3 & 566 & 64 & 1.0402 & 1 & 1 \\ \hline
\end{tabular}
}
\caption{Sperimentazione - Tipologia I - Random 5 nodi}
\label{tab:sperimentazione-tipo1-5nodi}
\end{table}


\subsection{Sperimentazione - Tipologia I - Random 7 nodi}

\begin{table}[H]
\centering
\scalebox{0.9} {
\begin{tabular}{|c|c|c|c|c|c|c|c|c|c|c|}
\hline
\textbf{k} & \textbf{p} & \textbf{a} & \textbf{o\_usw} & \textbf{o\_esw} & \textbf{n\_step} & \textbf{n\_usw} & \textbf{n\_esw} & \textbf{u\_poa} & \textbf{e\_poa} & \textbf{t} \\ \hline
3 & 0.36 & 9 & 1359 & 75 & 4 & 1359 & 60 & 1 & 1.25 & 2 \\ \hline
3 & 0.75 & 13 & 1922 & 206 & 5 & 1723 & 158 & 1.1154 & 1.3037 & 5 \\ \hline
3 & 0.54 & 11 & 1819 & 118 & 5 & 1694 & 118 & 1.0737 & 1 & 5 \\ \hline
2 & 0.89 & 18 & 1414 & 161 & 7 & 1361 & 117 & 1.0389 & 1.3760 & 2 \\ \hline
3 & 0.27 & 8 & 1456 & 151 & 7 & 1323 & 117 & 1.1005 & 1.2905 & 2 \\ \hline
\end{tabular}
}
\caption{Sperimentazione - Tipologia I - Random 7 nodi}
\label{tab:sperimentazione-tipo1-7nodi}
\end{table}


\subsection{Sperimentazione - Tipologia I - Random 10 nodi}

\begin{table}[H]
\centering
\scalebox{0.9} {
\begin{tabular}{|c|c|c|c|c|c|c|c|c|c|c|}
\hline
\textbf{k} & \textbf{p} & \textbf{a} & \textbf{o\_usw} & \textbf{o\_esw} & \textbf{n\_step} & \textbf{n\_usw} & \textbf{n\_esw} & \textbf{u\_poa} & \textbf{e\_poa} & \textbf{t} \\ \hline
2 & 0.77 & 41 & 3294 & 260 & 8 & 3155 & 241 & 1.0440 & 1.0728 & 2 \\ \hline
2 & 0.42 & 20 & 2281 & 127 & 7 & 2049 & 127 & 1.1132 & 1 & 2 \\ \hline
2 & 0.26 & 13 & 1786 & 77 & 6 & 1682 & 65 & 1.0618 & 1.1846 & 2 \\ \hline
2 & 0.99 & 45 & 3228 & 277 & 11 & 2935 & 247 & 1.0998 & 1.1214 & 2 \\ \hline
2 & 0.52 & 19 & 1842 & 108 & 7 & 1735 & 82 & 1.0616 & 1.3170 & 2 \\ \hline
\end{tabular}
}
\caption{Sperimentazione - Tipologia I - Random 10 nodi}
\label{tab:sperimentazione-tipo1-10nodi}
\end{table}

\section{Risultati sperimentazione - Tipologia II}
\justify
Qui di seguito vengono presentati i risultati relativi alla sperimentazione di Tipologia II.\\

\textbf{LEGGENDA : }

\begin{itemize}
	\item Il parametro \textbf{k} descrive il numero di colori relativo all'istanza corrente
	\item Il parametro \textbf{p} descrive la probabilità relativa alla densità dell'istanza corrente
	\item Il parametro \textbf{a} descrive il numero di archi dell'istanza corrente
	\item Il valore \textbf{n\_step} è il numero di step (mosse migliorative effettuate) ottenuto dall'applicazione, sull'istanza corrente, del calcolo della colorazione stabile seguendo la definizione di equilibrio di Nash
	\item Il valore \textbf{n\_usw} è il risultato del calcolo del benessere sociale utilitario relativo alla colorazione stabile trovata per l'istanza corrente
	\item Il valore \textbf{n\_esw} è il risultato del calcolo del benessere sociale egalitario relativo alla colorazione stabile trovata per l'istanza corrente
	\item Il parametro \textbf{t} è valore, assegnato dinamicamente, del limitatore temporale (1 = 1 minuto, 2 = 2 minuti, ecc...)
\end{itemize}

\subsection{Sperimentazione - Tipologia II - Random 15 nodi}

\begin{table}[H]
\centering
\begin{tabular}{|c|c|c|c|c|c|c|}
\hline
\textbf{k} & \textbf{p} & \textbf{a} & \textbf{n\_step} & \textbf{n\_usw} & \textbf{n\_esw} & \textbf{t} \\ \hline
7 & 0.75 & 78 & 9 & 9398 & 365 & 1 \\ \hline
7 & 0.84 & 83 & 19 & 9664 & 457 & 1 \\ \hline
13 & 0.49 & 50 & 17 & 6837 & 176 & 1 \\ \hline
5 & 0.30 & 35 & 7 & 5226 & 133 & 1 \\ \hline
9 & 0.87 & 90 & 19 & 10747 & 586 & 1 \\ \hline
5 & 0.49 & 41 & 15 & 5206 & 174 & 1 \\ \hline
9 & 0.89 & 97 & 14 & 10937 & 451 & 1 \\ \hline
10 & 0.79 & 84 & 19 & 10437 & 582 & 1 \\ \hline
4 & 0.88 & 87 & 9 & 9295 & 473 & 1 \\ \hline
13 & 0.38 & 34 & 14 & 4185 & 161 & 1 \\ \hline
\end{tabular}
\caption{Sperimentazione - Tipologia II - Random 15 nodi}
\label{tab:sperimentazione-tipo1-15nodi}
\end{table}

\subsection{Sperimentazione - Tipologia II - Random 30 nodi}

\begin{table}[H]
\centering
\begin{tabular}{|c|c|c|c|c|c|c|}
\hline
\textbf{k} & \textbf{p} & \textbf{a} & \textbf{n\_step} & \textbf{n\_usw} & \textbf{n\_esw} & \textbf{t} \\ \hline
29 & 0.09 & 48 & 30 & 7507 & 117 & 2 \\ \hline
3 & 0.86 & 366 & 40 & 32140 & 808 & 2 \\ \hline
11 & 0.15 & 72 & 36 & 9206 & 179 & 2 \\ \hline
16 & 0.03 & 12 & 14 & 3207 & 106 & 2 \\ \hline
8 & 0.15 & 70 & 37 & 10159 & 89 & 2 \\ \hline
6 & 0.20 & 77 & 35 & 10730 & 193 & 2 \\ \hline
6 & 0.36 & 151 & 38 & 15663 & 296 & 2 \\ \hline
3 & 0.39 & 168 & 19 & 15748 & 286 & 2 \\ \hline
28 & 0.56 & 248 & 34 & 26688 & 633 & 2 \\ \hline
17 & 0.42 & 191 & 46 & 22487 & 376 & 2 \\ \hline
\end{tabular}
\caption{Sperimentazione - Tipologia II - Random 30 nodi}
\label{tab:sperimentazione-tipo1-30nodi}
\end{table}

\subsection{Sperimentazione - Tipologia II - Random 45 nodi}

\begin{table}[H]
\centering
\begin{tabular}{|c|c|c|c|c|c|c|}
\hline
\textbf{k} & \textbf{p} & \textbf{a} & \textbf{n\_step} & \textbf{n\_usw} & \textbf{n\_esw} & \textbf{t} \\ \hline
11 & 0.69 & 666 & 74 & 70663 & 74 & 5 \\ \hline
19 & 0.82 & 810 & 56 & 86036 & 1522 & 5 \\ \hline
41 & 0.28 & 284 & 55 & 32495 & 351 & 5 \\ \hline
42 & 0.17 & 185 & 48 & 23126 & 225 & 5 \\ \hline
23 & 0.62 & 633 & 70 & 67407 & 953 & 5 \\ \hline
5 & 0.97 & 964 & 58 & 85815 & 1488 & 5 \\ \hline
25 & 0.23 & 205 & 48 & 25993 & 231 & 5 \\ \hline
29 & 0.58 & 557 & 57 & 58736 & 923 & 5 \\ \hline
9 & 0.49 & 481 & 62 & 51387 & 561 & 5 \\ \hline
15 & 0.32 & 338 & 52 & 37617 & 544 & 5 \\ \hline
\end{tabular}
\caption{Sperimentazione - Tipologia II - Random 45 nodi}
\label{tab:sperimentazione-tipo1-45nodi}
\end{table}

\subsection{Sperimentazione - Tipologia II - Random 60 nodi}

\begin{table}[H]
\centering
\begin{tabular}{|c|c|c|c|c|c|c|}
\hline
\textbf{k} & \textbf{p} & \textbf{a} & \textbf{n\_step} & \textbf{n\_usw} & \textbf{n\_esw} & \textbf{t} \\ \hline
30 & 0.27 & 456 & 67 & 51948 & 514 & 10 \\ \hline
52 & 0.05 & 52 & 56 & 13850 & 101 & 10 \\ \hline
13 & 0.44 & 783 & 83 & 85091 & 956 & 10 \\ \hline
51 & 0.84 & 1484 & 87 & 158991 & 2163 & 10 \\ \hline
38 & 0.77 & 1331 & 107 & 139927 & 1832 & 10 \\ \hline
11 & 0.90 & 1584 & 94 & 161166 & 2144 & 10 \\ \hline
32 & 0.66 & 1201 & 98 & 124281 & 1336 & 10 \\ \hline
3 & 0.23 & 415 & 65 & 40080 & 374 & 10 \\ \hline
14 & 0.39 & 711 & 92 & 76014 & 609 & 10 \\ \hline
47 & 0.86 & 1507 & 90 & 158109 & 2236 & 10 \\ \hline
\end{tabular}
\caption{Sperimentazione - Tipologia II - Random 60 nodi}
\label{tab:sperimentazione-tipo1-60nodi}
\end{table}

\subsection{Sperimentazione - Tipologia II - Random 75 nodi}

\begin{table}[H]
\centering
\begin{tabular}{|c|c|c|c|c|c|c|}
\hline
\textbf{k} & \textbf{p} & \textbf{a} & \textbf{n\_step} & \textbf{n\_usw} & \textbf{n\_esw} & \textbf{t} \\ \hline
14 & 0.04 & 107 & 80 & 17362 & 94 & 10 \\ \hline
24 & 0.89 & 2498 & 141 & 254295 & 2593 & 10 \\ \hline
23 & 0.23 & 703 & 99 & 79253 & 586 & 10 \\ \hline
2 & 0.35 & 960 & 44 & 64429 & 573 & 10 \\ \hline
41 & 0.07 & 207 & 74 & 28634 & 174 & 10 \\ \hline
10 & 0.33 & 918 & 95 & 99490 & 665 & 10 \\ \hline
73 & 0.10 & 310 & 77 & 39623 & 175 & 10 \\ \hline
16 & 0.14 & 400 & 96 & 46173 & 221 & 10 \\ \hline
8 & 0.08 & 221 & 85 & 28302 & 92 & 10 \\ \hline
6 & 0.29 & 803 & 115 & 85384 & 636 & 10 \\ \hline
\end{tabular}
\caption{Sperimentazione - Tipologia II - Random 75 nodi}
\label{tab:sperimentazione-tipo1-75nodi}
\end{table}