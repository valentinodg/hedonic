\documentclass{beamer}

\usepackage[utf8]{inputenc}
\usepackage[italian]{babel}

%\usepackage{graphicx}
%\graphicspath{ {./img/} }

\usetheme{Default} % Default Berkeley CambridgeUs Berlin
%\usecolortheme{beaver} % beaver seahorse

\setbeamercolor{structure}{bg=black, fg=white}
\setbeamercolor{block title alerted}{fg=white,bg=black}
\setbeamercolor{block body alerted}{fg=black, bg=gray!40}
\setbeamercolor{alerted text}{fg=red}
\setbeamercolor{background canvas}{bg=gray!10}


\title{Calcolo e performance di equilibri di Nash per il gioco della k-colorazione generalizzata}
\author{Valentino Di Giosaffatte \hfill Prof. Gianpiero Monaco}
\institute{Università degli Studi dell'Aquila}
\date{Anno Accademico 2017/2018}
%\logo{\includegraphics[height=1.3cm]{univaqlogowb.png}}

\begin{document}

\frame{\titlepage}

\begin{frame}

\frametitle{Obiettivi della sperimentazione}

\begin{itemize}
	\item Calcolo degli equilibri di Nash per il gioco della $k$-colorazione generalizzata
	\item Analisi delle performance dell'algoritmo per il calcolo delle soluzioni Nash-stabili effettuata attraverso la determinazione del numero di step relativi alle dinamiche di miglioramento
	\item Valutazione del benessere sociale utilitario e egalitario delle soluzioni Nash-stabili in relazione con il benessere sociale utilitario e egalitario delle soluzioni ottime, utilizzando le definizioni di prezzo dell'anarchia sperimentale utilitario e egalitario
\end{itemize}

\end{frame}

\begin{frame}

\frametitle{Frame teorema}
 
\begin{alertblock}{Important theorem}
Sample text in red box
\end{alertblock}
 
\end{frame}

\end{document}