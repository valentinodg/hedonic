\documentclass{beamer}

\usepackage[utf8]{inputenc}
\usepackage[italian]{babel}

\usepackage{dsfont}


%\usepackage{graphicx}
%\graphicspath{ {./img/} }

\usetheme{Default} % Default Berkeley CambridgeUs Berlin
%\usecolortheme{beaver} % beaver seahorse

\setbeamercolor{structure}{bg=black, fg=white}
\setbeamercolor{block title alerted}{fg=white,bg=black}
\setbeamercolor{block body alerted}{fg=black, bg=gray!40}
\setbeamercolor{alerted text}{fg=red}
\setbeamercolor{background canvas}{bg=gray!10}
\setbeamertemplate{itemize item}{\color{black}$\blacktriangleright$}


\title{Calcolo e performance di equilibri di Nash per il gioco della k-colorazione generalizzata}
\author{Valentino Di Giosaffatte \hfill Prof. Gianpiero Monaco}
\institute{Università degli Studi dell'Aquila}
\date{Anno Accademico 2017/2018}
%\logo{\includegraphics[height=1.3cm]{univaqlogowb.png}}


\begin{document}


\frame{\titlepage}


\begin{frame}
\frametitle{Obiettivi della sperimentazione}
\begin{itemize}
	\item Calcolo degli equilibri di Nash per il gioco della $k$-colorazione generalizzata
	\item Analisi delle performance dell'algoritmo per il calcolo delle soluzioni Nash-stabili effettuata attraverso la determinazione del numero di step relativi alle dinamiche di miglioramento
	\item Valutazione del benessere sociale utilitario e egalitario delle soluzioni Nash-stabili in relazione con il benessere sociale utilitario e egalitario delle soluzioni ottime, utilizzando le definizioni di prezzo dell'anarchia sperimentale utilitario e egalitario [descrive il prezzo dell'anarchia nel caso medio]
\end{itemize}
\end{frame}


\begin{frame}
\frametitle{Teoria dei giochi e giochi non-cooperativi}
La \alert{teoria dei giochi} è la disciplina scientifica che si occupa dello studio del comportamento e dei processi decisionali di soggetti razionali in un contesto di interdipendenza strategica. L'analisi è incentrata sugli scenari caratterizzati dalla presenza di situazioni di conflitto nelle quali gli attori sono costretti ad intraprendere strategie di cooperazione o competizione.\medskip

I \alert{giochi non-cooperativi} definiscono una specifica classe di giochi nella quale i giocatori non possono stipulare accordi vincolanti di cooperazione, anche normativamente.\medskip

Il criterio di comportamento razionale adottato nei giochi non-cooperativi è di carattere individuale ed è denominato \alert{strategia del massimo}. Tale definizione di razionalità va modellare il comportamento di un individuo intelligente e ottimista che si prefigge l'obiettivo di prendere sempre la decisione più vantaggiosa per se stesso.
\end{frame}


\begin{frame}
\frametitle{Equilibri di Nash}
L'equilibrio di Nash è una combinazione di strategie nella quale ciascun giocatore effettua la migliore scelta possibile, seguendo cioè una \alert{strategia dominante}, sulla base delle aspettative di scelta degli altri giocatori.\medskip

L'equilibrio di Nash rappresenta un \alert{concetto di soluzione} robusto per i giochi non-cooperativi.\medskip

L'equilibrio di Nash rappresenta inoltre una \alert{soluzione stabile}, poiché nessun giocatore ha interesse a deviare unilateralmente modificando la propria strategia.
\end{frame}


\begin{frame}
\frametitle{Definizione formale I}
\begin{itemize}
	\item Sia $G$ l'insieme dei \alert{giocatori}, che indicheremo con $i=1,\ldots,N$
	\item Sia $S$ l'insieme delle \alert{strategie}, costituito da un set di \(M\) vettori $S_{i}=\left(s_{{i,1}},s_{{i,2}},\ldots,s_{{i,j}},\ldots,s_{{i,M_{i}}}\right)$, ciascuno dei quali contiene l'insieme delle strategie che il giocatore \textit{i-esimo} ha a disposizione, cioè l'insieme delle azioni che esso può compiere (indichiamo con $s_i$ la strategia scelta dal giocatore $i$)
	\item Sia $U$ l'insieme delle \alert{funzioni} $u_{i}=U_{i}\left(s_{1},s_{2},\ldots,s_{i},\ldots,s_{N}\right)$ che associano ad ogni giocatore $i$ il guadagno (detto anche payoff) $u_i$ derivante da una data combinazione di strategie (il guadagno di un giocatore in generale non dipende solo dalla propria strategia ma anche dalle strategie scelte dagli avversari) 
\end{itemize}
\end{frame}


\begin{frame}
\frametitle{Definizione formale II}
\begin{itemize}
	\item Un \alert{equilibrio di Nash} per un dato gioco è una combinazione di strategie (che indichiamo con l'apice $e$) \[s_{1}^{e},s_{2}^{e},...,s_{N}^{e}\] tale che \[U_{i}\left(s_{1}^{e},s_{2}^{e},...,s_{i}^{e},...,s_{N}^{e}\right)\geq U_{i}\left(s_{1}^{e},s_{2}^{e},...,s_{i},...,s_{N}^{e}\right)\] $\forall i$ e $\forall s_i$ scelta dal giocatore \textit{i-esimo}.
\end{itemize}
\end{frame}


\begin{frame}
\frametitle{Descrizione del modello I}
\begin{itemize}
	\item Un'\alert{istanza} di gioco della $k$-colorazione generalizzata è una tupla $(G,K,P)$
	\item $G=(V,E,w)$ è un \alert{grafo} non-orientato pesato (senza self-loops e parallelismo degli archi), dove $|V|=n$, $|E|=m$ e $w : E\rightarrow\mathds{R}_{\geq 0}$ è la \alert{funzione} che associa un \alert{peso} intero positivo a ciascun arco [ogni $v \in V$ è un giocatore egoista]
	\item $K$ è un insieme di $k$ \alert{colori} disponibili (strategie), assumiamo $k\geq2$ [ciascun giocatore hai il medesimo set di azioni disponibili]
	\item Sia $P : V \times K \rightarrow \mathds{R}_{\geq 0}$ la \alert{funzione} che associa un \alert{profitto} a ciascun colore per ciascun giocatore [se il giocatore $v$ sceglie di usare il colore $i$ allora guadagna $P_v (i)$]
	\item Uno \alert{stato del gioco} è una $k$-colorazione $c = \{c_1,\ldots,c_n\}$ [$c_v$, con $1 \leq c_v \leq k$, è il colore scelto dal giocatore $v$ in $c$
\end{itemize}
\end{frame}


\begin{frame}
\frametitle{Descrizione del modello II}
\begin{itemize}
	\item Il \alert{payoff} (utilità) del giocatore $v$ nella colorazione $c$ è $\mu_c (v) = \sum_{u \in V:\{v, u\} \in E \wedge c_v \neq c_u} w(\{v, u\}) + P_v(c_v)$
	\item Data una colorazione $c$, una \alert{mossa migliorativa} del giocatore $v$ è una strategia $c'_v$ tale che $\mu_{(c_{-v}, c_v^{\prime})} (v) > \mu_c (v)$
	\item Una \alert{dinamica di miglioramento} è una sequenza di mosse migliorative
	\item Una colorazione $c$ è un \alert{equilibrio di Nash} se $\mu_c (v) \geq \mu_{(c_{-v}, c_v^{\prime})} (v)$
	\item \alert{Funzione di benessere sociale utilitario} : $SW_{UT} (c) = \sum_{v \in V} \mu_c (v) = \sum_{v \in V} P_v(c_v) + \sum_{\{v, u\} \in E : c_v \neq c_u} 2w(\{v, u\})$
	\item \alert{Funzione di benessere sociale egalitario} : $SW_{EG} (c) = min_{v \in V} \mu_c (v)$
	\item \alert{Prezzo dell'anarchia} : $PoA = \frac{max_{c \in C} SW(c)}{min_{c^{\prime} \in Q} SW(c^{\prime})}$
\end{itemize}
\end{frame}


\begin{frame}
\frametitle{Nozioni sul problema}
\begin{itemize}
	\item Il problema di calcolare un equilibrio di Nash su grafi non-orientati pesati è \alert{PLS-Completo}, anche per $k=2$, dato che il gioco del taglio massimo [Max-Cut Game] è un caso speciale del nostro gioco
	\item Se $k=2$ e i profitti sono impostati a $0$, otteniamo il \alert{gioco del taglio massimo}, celebre gioco PLS-Completo ampiamente trattato in letteratura
	\item Se i profitti sono impostati a $0$, otteniamo il \alert{gioco della $k$-colorazione}
\end{itemize}
\end{frame}


\begin{frame}
\frametitle{Risultati teorici}
	\begin{alertblock}{Proposizione 1}
		$\forall k$, ogni gioco della $k$-colorazione generalizzata $(G,K,P)$ finito è convergente
	\end{alertblock}
	\begin{alertblock}{Teorema 1}
		Il prezzo dell'anarchia per il gioco della $k$-colorazione generalizzata è al più $2$ 
	\end{alertblock}	
	\begin{alertblock}{Teorema 2}
		Il prezzo dell'anarchia utilitario per il gioco della $k$-colorazione generalizzata è almeno $2$, anche per il caso speciale di grafi stella non-pesati
	\end{alertblock}	
	\begin{alertblock}{Teorema 3}
		Il prezzo dell'anarchia egalitario per il gioco della $k$-colorazione generalizzata è $2$ 
	\end{alertblock}	
\end{frame}


\begin{frame}
\frametitle{Implementazione I}
\begin{itemize}
	\item Utilizzo del linguaggio \alert{Python} [Standard Library]
	\item Utilizzo della libreria di creazione e manipolazione di grafi \alert{NetworkX} [e altre minori]
	\item Costruzione dei moduli per la generazione e per la lettura asincrona di grafi [\textit{generator.py}, \textit{reader.py}]
	\item Utilizzo di \alert{strutture dati} efficenti come \textit{liste} e \textit{dizionari} sia in forma singola che innestata [\textit{single or nested list and dictionary comprehension}]
	\item Implementazione degli \alert{algoritmi per il calcolo dell'ottimo} con funzioni di benessere sociale utilitario e egalitario utilizzando una strategia incentrata sulla forza bruta [privi di tecniche di ottimizzazione delle iterazioni]
\end{itemize}
\end{frame}


\begin{frame}
\frametitle{Implementazione II}
\begin{itemize}
	\item Implementazione dell'\alert{algoritmo per il calcolo della colorazione stabile} seguendo la definizione di equilbrio di Nash utilizzando 3 importanti strategie di ottimizzazione
	\item Strategia per il calcolo della \textit{\alert{best move}} per ciascun nodo, in modo da minimizzare il valore relativo agli step totali effettuati dall'algoritmo durante la ricerca della dinamica [incremento della complessità computazionale]
	\item Doppia strategia per il salto delle iterazioni basata sul controllo dei colori e dei miglioramenti effettuati [abbattimento della complessità computazionale]
\end{itemize}
\end{frame}


\begin{frame}
\frametitle{Assunzioni generali}
Sperimentazione effettuata su \alert{grafi randomici} di tipo \textit{gnp\_random\_graph} (detti grafi di Erdős-Rényi o grafi binomiali) fissando il parametro \alert{$n$} [\alert{numero di nodi}] e fissando o facendo oscillare randomicamente i 2 parametri \alert{$k$} [\alert{numero di colori}, $2 \geq k \geq n$] e \alert{$p$} [\alert{probabilità di generare archi tra le coppie di nodi}, $0 \geq p \geq 1$]
\begin{itemize}
	\item \alert{Tipologia I} : calcolo degli ottimi con funzioni di benessere sociale utilitario e egalitario, calcolo dell'equilibrio di Nash, definizione del valore di benessere sociale utilitario e egalitario della colorazione stabile, definizione dei valori di prezzo dell'anarchia sperimentale utilitario e egalitario (su istanze piccole)
	\item \alert{Tipologia II} : calcolo dell'equilibrio di Nash, definizione del valore di benessere sociale utilitario e egalitario della colorazione stabile (su istanze medio-grandi)
\end{itemize}
\end{frame}


\begin{frame}
\frametitle{Tipologia I [$n=5$, oscillazione random di $k$ e $p$]}
\begin{table}[H]
\centering
\scalebox{0.9} {
\begin{tabular}{|c|c|c|c|c|c|c|c|c|c|c|}
\hline
\multicolumn{3}{|c|}{\textbf{}} & \multicolumn{2}{c|}{\textbf{\color{orange}opt}} & \multicolumn{3}{c|}{\textbf{\color{orange}nash}} & \multicolumn{2}{c|}{\textbf{\color{orange}poa}} & \textbf{} \\ \hline
\textbf{\alert{k}} & \textbf{\alert{p}} & \textbf{a} & \textbf{util} & \textbf{egal} & \textbf{step} & \textbf{util} & \textbf{egal} & \textbf{util} & \textbf{egal} & \textbf{t} \\ \hline
5 & 0.90 & 9 & 314 & 45 & 5 & 314 & 41 & 1 & 1.0975 & 1 \\ \hline
4 & 0.27 & 5 & 485 & 94 & 4 & 485 & 94 & 1 & 1 & 1 \\ \hline
3 & 0.82 & 9 & 410 & 45 & 4 & 410 & 40 & 1 & 1.1250 & 1 \\ \hline
4 & 0.14 & 4 & 394 & 40 & 4 & 394 & 40 & 1 & 1 & 1 \\ \hline
2 & 0.79 & 7 & 420 & 37 & 5 & 420 & 37 & 1 & 1 & 1 \\ \hline
5 & 0.65 & 9 & 497 & 86 & 6 & 497 & 86 & 1 & 1 & 2 \\ \hline
3 & 0.40 & 6 & 805 & 120 & 5 & 766 & 120 & 1.0509 & 1 & 1 \\ \hline
3 & 0.52 & 5 & 932 & 85 & 4 & 932 & 85 & 1 & 1 & 1 \\ \hline
5 & 1 & 10 & 913 & 149 & 3 & 913 & 149 & 1 & 1 & 2 \\ \hline
2 & 0.24 & 3 & 590 & 64 & 3 & 566 & 64 & 1.0402 & 1 & 1 \\ \hline
\end{tabular}
}
\end{table}
\end{frame}


\begin{frame}
\frametitle{Tipologia I [$n=5$, $k=3$, random $0.40 \geq p \geq 1$] e [$n=5$, $p=0.70$, random di $2 \geq k \geq 5$]}
\begin{table}[H]
\centering
\scalebox{0.9} {
\begin{tabular}{|c|c|c|c|c|c|c|c|c|c|c|}
\hline
\multicolumn{2}{|c|}{\textbf{}} & \multicolumn{2}{c|}{\textbf{\color{orange}opt}} & \multicolumn{3}{c|}{\textbf{\color{orange}nash}} & \multicolumn{2}{c|}{\textbf{\color{orange}poa}} & \multicolumn{1}{c|}{\textbf{}} \\ \hline
\textbf{\alert{p}} & \textbf{a} & \textbf{util} & \textbf{egal} & \textbf{step} & \textbf{util} & \textbf{egal} & \textbf{util} & \textbf{egail} & \textbf{t} \\ \hline
0.40 & 5 & 1064 & 148 & 6 & 1064 & 126 & 1 & 1.1746 & 1 \\ \hline
0.50 & 7 & 1113 & 100 & 3 & 1019 & 76 & 1.0922 & 1.3157 & 1 \\ \hline
0.60 & 6 & 1253 & 163 & 8 & 1121 & 163 & 1.1177 & 1 & 1 \\ \hline
0.70 & 5 & 862 & 95 & 7 & 862 & 95 & 1 & 1 & 1 \\ \hline
0.80 & 9 & 1104 & 166 & 9 & 1061 & 125 & 1.0405 & 1.328 & 1 \\ \hline
0.90 & 9 & 1249 & 210 & 7 & 1249 & 173 & 1 & 1.1475 & 1 \\ \hline
1 & 10 & 1564 & 261 & 3 & 1451 & 246 & 1.0778 & 1.0609 & 1 \\ \hline
\end{tabular}
}
\end{table}
\begin{table}[H]
\centering
\scalebox{0.9} {
\begin{tabular}{|c|c|c|c|c|c|c|c|c|c|c|}
\hline
\multicolumn{2}{|c|}{\textbf{}} & \multicolumn{2}{c|}{\textbf{\color{orange}opt}} & \multicolumn{3}{c|}{\textbf{\color{orange}nash}} & \multicolumn{2}{c|}{\textbf{\color{orange}poa}} & \multicolumn{1}{c|}{\textbf{}} \\ \hline
\textbf{\alert{k}} & \textbf{a} & \textbf{util} & \textbf{egal} & \textbf{step} & \textbf{util} & \textbf{egal} & \textbf{util} & \textbf{egail} & \textbf{t} \\ \hline
2 & 6 & 794 & 118 & 3 & 778 & 77 & 1.0205 & 1.5324 & 1 \\ \hline
3 & 6 & 930 & 148 & 6 & 912 & 120 & 1.0197 & 1.2333 & 1 \\ \hline
4 & 6 & 1015 & 152 & 6 & 1015 & 152 & 1 & 1 & 1 \\ \hline
5 & 6 & 1063 & 149 & 4 & 1063 & 149 & 1 & 1 & 1 \\ \hline
\end{tabular}
}
\end{table}
\end{frame}


\begin{frame}
\frametitle{Tipologia II [$n=15$, oscillazione random di $k$ e $p$]}
\begin{table}[H]
\centering
\scalebox{1} {
\begin{tabular}{|c|c|c|c|c|c|c|}
\hline
\multicolumn{3}{|c|}{\textbf{}} & \multicolumn{3}{c|}{\textbf{\color{orange}nash}} & \textbf{} \\ \hline
\textbf{\alert{k}} & \textbf{\alert{p}} & \textbf{a} & \textbf{step} & \textbf{utilitarian} & \textbf{egalitarian} & \textbf{t} \\ \hline
7 & 0.75 & 78 & 9 & 9398 & 365 & 1 \\ \hline
7 & 0.84 & 83 & 19 & 9664 & 457 & 1 \\ \hline
13 & 0.49 & 50 & 17 & 6837 & 176 & 1 \\ \hline
5 & 0.30 & 35 & 7 & 5226 & 133 & 1 \\ \hline
9 & 0.87 & 90 & 19 & 10747 & 586 & 1 \\ \hline
5 & 0.49 & 41 & 15 & 5206 & 174 & 1 \\ \hline
9 & 0.89 & 97 & 14 & 10937 & 451 & 1 \\ \hline
10 & 0.79 & 84 & 19 & 10437 & 582 & 1 \\ \hline
4 & 0.88 & 87 & 9 & 9295 & 473 & 1 \\ \hline
13 & 0.38 & 34 & 14 & 4185 & 161 & 1 \\ \hline
\end{tabular}
}
\end{table}
\end{frame}


\begin{frame}
\frametitle{Sperimentazione [$n=15$, $k=8$, random $0.40 \geq p \geq 1$]}
\begin{table}[H]
\centering
\scalebox{1} {
\begin{tabular}{|c|c|c|c|c|c|}
\hline
\multicolumn{3}{|c|}{\textbf{}} & \multicolumn{2}{c|}{\textbf{\color{orange}nash}} & \textbf{} \\ \hline
\textbf{\alert{p}} & \textbf{a} & \textbf{step} & \textbf{utilitarian} & \textbf{egalitarian} & \textbf{t} \\ \hline
0.40 & 53 & 18 & 6199 & 213 & 1 \\ \hline
0.50 & 54 & 25 & 7243 & 338 & 1 \\ \hline
0.60 & 69 & 22 & 8747 & 459 & 1 \\ \hline
0.70 & 82 & 21 & 9376 & 414 & 1 \\ \hline
0.80 & 86 & 16 & 10292 & 556 & 1 \\ \hline
0.90 & 93 & 22 & 11063 & 569 & 1 \\ \hline
1 & 105 & 17 & 12055 & 604 & 1 \\ \hline
\end{tabular}
}
\end{table}
\end{frame}


\begin{frame}
\frametitle{Sperimentazione [$n=15$, $p=0.70$, random $2 \geq k \geq 15$]}
\end{frame}


\begin{frame}
\frametitle{Lavori futuri}
\begin{itemize}
	\item Sviluppo di implementazioni più efficaci per ciò che concerne il calcolo degli ottimi con funzioni di benessere sociale utilitario e egalitario
	\item Potrebbero essere utilizzate tecniche di ricerca operativa (ad esempio, mirate a ridurre lo spazio di ricerca dell'ottimo) per abbattere la complessità computazionale derivante dall'approccio a forza bruta
	\item Tale miglioria garantirebbe la possibilità di effettuare uno studio più approfondito riguardante la variazione del prezzo dell'anarchia sperimentale utilitario e egalitario per istanze maggiori di quelle analizzate, con esecuzioni effettuate in tempi ragionevoli
\end{itemize}
\end{frame}











\end{document}